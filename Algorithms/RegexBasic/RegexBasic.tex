\lab{Algorithms}{Basic Regular Expressions}{Basic Regular Expressions}
\objective{Learn the basics of using regular expressions to find text}
\label{lab:Regex_Basic}

Regular expressions are a tool that allow for quick searching and replacing of general patterns of text.
While nearly all text editors have a feature that will find and replace exact strings of text, regular expressions are used to find text in a much more general way. For example, using a single regular expression, you can find every email address in a text file without having to sift through it by hand.
% include email regex?
% TODO add a cool example here... this will do for now:
    % As an example, the following program will find and fill in missing colons after control statements in normally formatted python files: 
    % % ^((def|if|elif|else|while|for|with).*):\s*$

\section*{Terminology and Basics}
A ``regular expression'' is basically just a string of characters that follow a certain syntax. Computer programs can then interpret these expressions as instructions to search for certain kinds of text. We will often call regular expressions ``patterns'', and we will say that certain patterns ``match'' certain strings. The general idea is that a regular expression represents a large set of strings (for example, all valid email addresses), and if a specific string is in that set, we say that the regular expression matches that string.
% TODO talk w. ryan about what they'll know at this point about DFSMs

\begin{warn} %TODO why doesn't this work?
Regular expression libraries have been implemented and are a part of the standard distribution of nearly every programming language, and many text editors have a find-and-replace mode that uses regular expressions. Unfortunately, the syntax for regular expressions may be slightly different in each implementation. There is no universal standard for all regular expressions across all platforms. However, the orginal syntax and a few variants are very widespread, so the basic regular expression techniques we learn in this lab should be virtually the same in almost every situation you will encounter them.
\end{warn}

The simplest use of regular expressions is to match text literally.
For example, the pattern \li{"cat"} matches the string \li{"cat"} but does not match the strings \li{"dog"} or \li{"bat"}.

Now that we have a general idea of what regular expressions are, we will see how to use them in Python.

\section*{Regular Expressions in Python}
The python package \li{re} contains the functionality for using regular expressions.
To use it, simply run the command \li{import re}.

The following Python code demonstrates what we said earlier about the regular expression \li{"cat"}:
\begin{lstlisting}
>>> bool(re.match("cat", "cat"))
True
>>> bool(re.match("cat", "dog"))
False
>>> bool(re.match("cat", "bat"))
False
\end{lstlisting}

The main functions we will use are \li{re.match(pattern, string_to_test)} and \li{re.compile(pattern)}.
You can think of \li{re.match} as returning a boolean value representing whether the given pattern matched the given string.
The function \li{re.compile} returns a compiled object that represents a regular expression. You can then call the \li{match} function on this compiled object to get a boolean value.

The following code shows an example of how to use \li{re.compile}:
\begin{lstlisting}
>>> pattern = re.compile("any regular expression")
>>> result = pattern.match("any string")
\end{lstlisting}
The above code is equivalent to the following:
\begin{lstlisting}
>>> result = re.match("any regular expression", "any string")
\end{lstlisting}
Most programs use the compiled form (the first of the above two examples) for efficiency.

When constructing a regular expression, it is best to construct your pattern string using Python's syntax for raw strings by prefacing the string with the \li{'r'} character. This causes the constructed string to treat backslashes as actual backslash characters, rather than the start of an escape sequence.

For example:
\begin{lstlisting}
>>> normal = "hello\nworld"
>>> raw = r"hello\nworld"
>>> print normal
hello
world
>>> print raw
hello\nworld
>>> type(normal), normal
(str, 'hello\nworld')
>>> type(raw), raw
(str, 'hello\\nworld')
\end{lstlisting}
Note that \li{raw} and \li{normal} are both python strings; one was just constructed differently. Also notice that when we constructed \li{raw}, it inserted an extra backslash before the existing backslash.

We use raw strings because the backslash character is a very important special character in regular expressions. If we wanted to use backslash characters as part of a normally-constructed Python string, we would need to either escape every single backslash by using two backslashes each time, or we could take the much easier and less confusing route of using Python's raw strings.
To demonstrate this effect, suppose we wanted to know whether the regular expression \li{"\\\$3\\.00"} matched the string \li{"\$3.00"}. We could get our answer in either of the following ways:
\begin{lstlisting}
>>> bool(re.match("\\$3\\.00", "$3.00"))
True
>>> bool(re.match(r"\$3\.00", "$3.00"))
True
\end{lstlisting}
(You will see why this pattern matches this string soon)

Remember, readability counts.

\section*{Literal Characters and Metacharacters}
The following characters are used as metacharacters in regular expressions:
\begin{lstlisting}
. ^ $ * + ? { } [ ] \ | ( )
\end{lstlisting}
These characters mean special things when used in regular expressions, making the vast power of regular expressions possible.
We will get to using these characters later. For now, what do we do if want to match these characters literally?
We simply escape these characters using the metacharacter \li{'\\'}:
\begin{lstlisting}
>>> pattern = re.compile(r"\$2\.95, please")
>>> bool(pattern.match("$2.95, please"))
True
>>> bool(pattern.match("$295, please"))
False
>>> bool(pattern.match("$2.95"))
False
\end{lstlisting}
\begin{problem}
Define the variable \li{pattern_string} using literal characters and escaped metacharacters in such a way that the following python program prints \li{True}:
\begin{lstlisting}
import re
pattern_string = r""  # Edit this line
pattern = re.compile(pattern_string)
print bool(pattern.match("^{(!%.*_)}&"))
\end{lstlisting}
\end{problem}

A little misleadingly, the \li{re.match} method isn't actually checking whether the given regular expression matches entire strings.
Rather, it checks whether the regular expression matches \emph{at} the \emph{beginning} of the string, even if the string continues on afterward.
For example:
% \regex   x   x xabc abcx
\begin{lstlisting}
>>> pattern = re.compile(r"x")
>>> bool(pattern.match("x"))
True
>>> bool(pattern.match("xabc"))
True
>>> bool(pattern.match("abcx"))
False
\end{lstlisting}

You might not expect the pattern \li{'x'} to match the string \li{"xabc"}, but it does.
This can cause confusion and headache, so we'll have to be a little more precise with the help of metacharacters.

The \emph{line anchor} metacharacters, \li{'^'} and \li{'\$'}, are used to match the start and the end of a line of text, respectively.
Let's see them in action:
% \regex   ^x$   x xabc abcx
\begin{lstlisting}
>>> pattern = re.compile(r"^x$")
>>> bool(pattern.match("x"))
True
>>> bool(pattern.match("xabc"))
False
>>> bool(pattern.match("abcx"))
False
\end{lstlisting}
That's much better.

% TODO should i include this?
% \begin{Warn}
%     From now until we say otherwise, we will be only using single-line regular expressions. At that point, certain things change slightly, and instead of confusing you with them now, we'll save those differnces until then.
% \end{Warn}

Let's move on to \li{'('}, \li{')'}, and \li{'|'}.
The \li{'|'} character (the ``pipe'' character, usually found on the key below the backspace key) matches one of two or more regular expressions:
\begin{lstlisting}
>>> pattern2 = re.compile(r"^red$|^blue$")
>>> pattern3 = re.compile(r"^red$|^blue$|^orange$")
>>> bool(pattern2.match("red")),     bool(pattern3.match("red"))
(True, True)
>>> bool(pattern2.match("blue")),    bool(pattern3.match("blue"))
(True, True)
>>> bool(pattern2.match("orange")),  bool(pattern3.match("orange"))
(False, True)
>>> bool(pattern2.match("redblue")), bool(pattern3.match("redblue"))
(False, False)
\end{lstlisting}

You can think of \li{'|'} as doing an ``or'' operation.
How would we create a regular expression that matched both \li{"one fish"} and \li{"two fish"}?
Although the regular expression \li{"one fish|two fish"} works, there is a better way, by using both the pipe character and parentheses:
\begin{lstlisting}
>>> pattern = re.compile(r"^(one|two) fish$")
>>> bool(pattern.match("one fish"))
True
>>> bool(pattern.match("two fish"))
True
>>> bool(pattern.match("three fish"))
False
>>> bool(pattern.match("one two fish"))
False
\end{lstlisting}
% assert (bool(pattern.match("one fish")), bool(pattern.match("two fish")), bool(pattern.match("three fish")), bool(pattern.match("one two fish"))) == (1,1,0,0)
As the above example demonstrates, parentheses are used to group sequences of characters together and change the order of precedence of the metacharacters, much like how parentheses work in an arithmetic expression such as \li{3*(4+5)}. In regular expressions, the \li{'|'} metacharacter has the lowest precedence out of all the metacharacters.

Parentheses actually have more uses, which we will learn later.
For now, note that parentheses aren't matched literally:
% \regex   r(hi)no(c(e)ro)s   rhinoceros
\begin{lstlisting}
>>> bool(re.match(r"r(hi)no(c(e)ro)s", "rhinoceros"))
True
\end{lstlisting}

\begin{problem}
Give a description of \emph{every} string that the pattern \li{"^one|two fish\$"} matches. Hint: Consider the strings \li{"one more"} and \li{"thirty-two fish"}.
\end{problem}

\begin{problem}
Define the variable \li{pattern_string} using the metacharacter \li{'|'} and parentheses in such a way that the following python program prints \li{True}:

\begin{lstlisting}
import re
pattern_string = r""  # Edit this line
pattern = re.compile(pattern_string)
strings_to_match = [ "Book store", "Book supplier", "Mattress store", "Mattress supplier", "Grocery store", "Grocery supplier"]
print all(pattern.match(string) for string in strings_to_match)
\end{lstlisting}
Your regular expression should not match any other string, including strings such as \li{"Book store sale"}.
\end{problem}

\section*{Character Classes}
The metacharacters \li{'['} and \li{']'} are used to create \emph{character classes}.
Here they are in action:
% \regex   b[aei]d   bad bed bid bud bod byd
\begin{lstlisting}
>>> pattern = re.compile(r"[xy]")
>>> bool(pattern.match("x"))
True
>>> bool(pattern.match("y"))
True
>>> bool(pattern.match("z"))
False
>>> bool(pattern.match("x: Why does this match? Were you paying attention?"))
True
\end{lstlisting}

In essence, a character class will match any one out of several characters.

Inside character classes, there are two additional metacharacters: \li{'-'} and \li{'^'}.
Although we've already seen \li{'^'} as a metacharacter, it has a different meaning when used inside a character class.
When \li{'^'} appears \emph{as the first character} in a character class, the character class matches anything not specified instead.
Think of \li{'^'} as performing a set complement operation on the character class.
For example:
% \regex   ^[^ab]$   x # a b
\begin{lstlisting}
>>> pattern = re.compile(r"^[^ab]$")
>>> bool(pattern.match("x"))
True
>>> bool(pattern.match("#"))
True
>>> bool(pattern.match("a"))
False
>>> bool(pattern.match("b"))
False
\end{lstlisting}

Note that the two \li{'^'} characters mean completely different things; the first one anchors us at the beginning of the line, while the second \li{'^'} performs a set complement operation on the character class \li{"[ab]"}.

The other character class metacharacter is \li{'-'}. This is used to specifiy a range of values.
For example:
% \regex   ^[a-z][0-9][0-9]$   a90 z73 A90 zs3
\begin{lstlisting}
>>> pattern = re.compile(r"^[a-z][0-9][0-9]$")
>>> bool(pattern.match("a90"))
True
>>> bool(pattern.match("z73"))
True
>>> bool(pattern.match("A90"))
False
>>> bool(pattern.match("zs3"))
False
\end{lstlisting}

Multiple ranges or characters can be included in a single character class; in this case, the character class will match any character that fits either criterion:
\begin{lstlisting}
>>> pattern = re.compile(r"^[abcA-C][0-27-9]$")
>>> bool(pattern.match("b8"))
True
>>> bool(pattern.match("B2"))
True
>>> bool(pattern.match("a9"))
True
>>> bool(pattern.match("a4"))
False
>>> bool(pattern.match("E1"))
False
\end{lstlisting}

Finally, there are some built-in shorthands for certain character classes: 
\begin{itemize}
    \item \li{'\\d'} (think ``digit'') matches any digit. It is equivalent to \li{"[0-9]"}.
    \item \li{'\\w'} (think ``word'') matches any alphanumeric character or underscore. It is equivalent to \li{"[a-zA-Z0-9_]"}.
    \item \li{'\\s'} (think ``space'') matches any whitespace character. It is equivalent to \li{"[ \t\n\r\f\v]"}.
\end{itemize}
The following character classes are the complements of those above:
\begin{itemize}
    \item \li{'\\D'} is equivalent to \li{"[\^0-9]"} or \li{"[\^\D]"}
    \item \li{'\\W'} is equivalent to \li{"[\^a-zA-Z0-9\_]"} or \li{"[\^\W]"}
    \item \li{'\\S'} is equivalent to \li{"[^ \\t\\n\\r\\f\\v]"} or \li{"[\^\S]"}
\end{itemize}

These character classes can be used in character classes; for example, \li{"[\_A-Z\\s]"} will match an underscore, any capital letter, or any whitespace character.

The \li{'.'} metacharacter, equivalent to \li{"[\^\\n]"} on UNIX and \li{"[\^\\r\\n]"} on Windows, matches any character except for a line break.
For example:
% \regex   ^.\d.$   a0b 888 n2% abc m&m cat
\begin{lstlisting}
>>> pattern = re.compile(r"^.\d.$")
>>> bool(pattern.match("a0b"))
True
>>> bool(pattern.match("888"))
True
>>> bool(pattern.match("n2%"))
True
>>> bool(pattern.match("abc"))
False
>>> bool(pattern.match("m&m"))
False
>>> bool(pattern.match("cat"))
False
\end{lstlisting}

\begin{problem}
Define the variable \li{pattern_string} in such a way that the following python program prints \li{True}:

\begin{lstlisting}
import re
pattern_string = r""  # Edit this line
pattern = re.compile(pattern_string)

strings_to_match = ["a", "b", "c", "x", "y", "z"]
uses_line_anchors = (pattern_string.startswith('^') and pattern_string.endswith('$'))
solution_is_clever = (len(pattern_string) == 8)
matches_list = all(pattern.match(string) for string in strings_to_match)

print uses_line_anchors and solution_is_clever and matches_list
\end{lstlisting}
\end{problem}

\begin{problem}
A \emph{valid python identifier} is defined as any string composed of any alphabetic character or underscore, followed by any (possibly empty) sequence of alphanumeric characters and underscores.

Define the variable \li{identifier_pattern_string} that defines a regular expression that matches valid python identifiers that are exactly five characters long. 

To help you test your pattern, the following program should print \li{True}. (This is necessary but not sufficient to show your regular expression is correct):
\begin{lstlisting}
import re
identifier_pattern_string = r""  # Edit this line
identifier_pattern = re.compile(identifier_pattern)

valid = ["mouse", "HORSE", "_1234", "__x__", "while"]
not_valid = ["3rats", "err*r", "sq(x)", "too_long"]

print all(identifier_pattern.match(string) for string in valid) and not any(identifier_pattern.match(string) for string in not_valid)
\end{lstlisting}

Hint: Use the \li{'\\w'} character class to keep your regular expression relatively short.
\begin{info}
As you might have noticed, using this definition, \li{"while"} is considered a valid python identifier, even though it really is a reserved word. In the following problems, we will make a few other simplifying assumptions about the python language.
\end{info}
\end{problem}

\section*{Repetition}
The metacharacters \li{'*'}, \li{'+'}, \li{'\{'}, and \li{'\}'} are used for repetition.

The \li{'*'} metacharacter means ``Match zero or more times (as many as possible)'' when it follows another regular expression.
For instance:
% \regex   ^a*b$   b ab aab aaab abab abc
\begin{lstlisting}
>>> pattern = re.compile(r"^a*b$")
>>> bool(pattern.match("b"))
True
>>> bool(pattern.match("ab"))
True
>>> bool(pattern.match("aab"))
True
>>> bool(pattern.match("aaab"))
True
>>> bool(pattern.match("abab"))
False
>>> bool(pattern.match("abc"))
False
\end{lstlisting}

The \li{'+'} metacharacter means ``Match one or more times (as many as possible)'' when it follows another regular expression.
As an example:
% \regex   ^h[ia]+$   ha hii hiai haaia h hah he
\begin{lstlisting}
>>> pattern = re.compile(r"^h[ia]+$")
>>> bool(pattern.match("ha"))
True
>>> bool(pattern.match("hii"))
True
>>> bool(pattern.match("hiaiaa"))
True
>>> bool(pattern.match("h"))
False
>>> bool(pattern.match("hah"))
False
\end{lstlisting}

It's important to understand why \li{"hiaiaa"} \emph{is} a match here; matching multiple times means matching the preceeding \emph{expression} multiple times, not matching the \emph{results} of the preceeding expression multiple times. We haven't yet learned how to construct a regular expression with that behavior.

The \li{'?'} metacharacter means ``Match one time (if possible) or do nothing (i.e. match zero times)'' when it follows another regular expression:
% \regex   ^abc?$   abc ab abd ac
\begin{lstlisting}
>>> pattern = re.compile(r"^abc?$")
>>> bool(pattern.match("abc"))
True
>>> bool(pattern.match("ab"))
True
>>> bool(pattern.match("abd"))
False
>>> bool(pattern.match("ac"))
False
\end{lstlisting}

The curly brace metacharacters are used to specify a more precise amount of repetition:
% \regex   ^a{2,4}$   a aa aaa aaaa aaaaa aaaaaa bbb
\begin{lstlisting}
>>> pattern = re.compile(r"^a{2,4}$")
>>> bool(pattern.match("a"))
False
>>> bool(pattern.match("aa"))
True
>>> bool(pattern.match("aaa"))
True
>>> bool(pattern.match("aaaa"))
True
>>> bool(pattern.match("aaaaa"))
False
\end{lstlisting}

If two arguments \li{x} and \li{y} are given to the curly braces (i.e., \li{\{x, y\}}), the preceeding regular expression must appear between \li{x} and \li{y} times, inclusive, in order for the overall expression to match.

\begin{warn}
In this last example, line anchors can save us from a lot of confusion. Note the differences between the following example and the example immediately above:
% \regex   a{2,4}   a aa aaa aaaa aaaaa aaaaaa
\begin{lstlisting}
>>> pattern = re.compile(r"a{2,4}")
>>> bool(pattern.match("a"))
False
>>> bool(pattern.match("aa"))
True
>>> bool(pattern.match("aaa"))
True
>>> bool(pattern.match("aaaa"))
True
>>> bool(pattern.match("aaaaa"))
True
\end{lstlisting}
\end{warn}

If only one argument \li{x} is given and is followed by a comma, the preceeding regular expression must match \li{x} or more times. If only one argument \li{x} is given without a comma, the preceeding regular expression must match \emph{exactly} \li{x} times.
For example:
\begin{lstlisting}
>>> exactly_three = re.compile(r"^a{3}$")
>>> three_or_more = re.compile(r"^a{3,}$")
>>> def test_both_patterns(string):
...     return bool(exactly_three.match(string)), bool(three_or_more.match(string))
>>> test_both_patterns("a")
(False, False)
>>> test_both_patterns("aa")
(False, False)
>>> test_both_patterns("aaa")
(True, True)
>>> test_both_patterns("aaaa")
(False, True)
>>> test_both_patterns("aaaaa")
(False, True)
\end{lstlisting}

\begin{problem}
Modify your definition of \li{identifier_pattern_string} from the previous problem to match valid python identifiers of any length.
\end{problem}

% For example, your regular expression should match each of the following: % todo format
%     compile(),
%     \_\_unicode\_\_(), and
%     a113()
% and should not match any of the following:
%     pattern.match(),
%     fxn(no\_arguments\_please),
%     365\_day\_timer(),
%     variable,
%     err*r(), and
%     sleep()()

\begin{problem}
A \emph{valid python parameter definition} is defined as the concatenation of the following strings:
\begin{itemize}
    \item any valid python identifier
    \item any number of spaces
    \item (optional) an equals sign, any number of spaces, and either any real number, a single quote followed by any number of non-single-quote characters followed by a single quote, or any valid python identifier
\end{itemize}

Define a variable \li{parameter_pattern_string} that defines a regular expression that matches valid python parameter definitions.

For example, each element of \li{["max=4.2", "string= ''", "num_guesses", "msg ='\\\\'", "volume_fn=_CALC_VOLUME"]} is a valid python parameter definition, while each element of \li{["300", "no spaces", "is_4=(value==4)", "pattern = r'^one|two fish$'", 'string="these last two are actually valid in python, but they should not be matched by your pattern"']} is not. % TODO add more negative examples maybe?
\end{problem}

\begin{problem}
A \emph{valid python function definition} is defined as the concatenation of the following strings:
\begin{itemize}
    \item \li{"def "}
    \item any valid python identifier
    \item \li{"("}
    \item a squence of any number of (possibly zero) valid python parameter definitions, separated by any number of spaces followed by a comma followed by any number of spaces
    \item \li{")"}
    \item \li{":"}
\end{itemize}
with any number of spaces between each element of the above list.

Define a variable \li{function_pattern_string} that defines a regular expression that matches valid python function definitions. Using 
\li{function_pattern_string}, write a program \texttt{match\_function\_definition.py}
which repeatedly asks the user for a string and prints out whether that string is a valid python function definition, halting only if the inputted string is \li{"exit"}.

For example, the program should behave as follows:
\begin{lstlisting}
>>> run match_function_definition.py
Enter a string>>> def compile(pattern,string):
True
Enter a string>>> def  space  ( ) :
True
Enter a string>>> def func(_dir, file_path='\Desktop\files', val=_PI):
True
Enter a string>>> def func(num=3., num=.5, num=0.0):
True
Enter a string>>> def func(num=., error,):
False
Enter a string>>> def variable:
False
Enter a string>>> def not.allowed(, *args):
False
Enter a string>>> def err*r('no parameter name'):
False
Enter a string>>> def func(value=_MY_CONSTANT, msg='%s' % _DEFAULT_MSG):
False
Enter a string>>> def func(s1='', a little tricky, s2=''):
False
Enter a string>>> def func(): Remember your line anchors!
False
Enter a string>>> deffunc()
False
Enter a string>>> func():
False
Enter a string>>> exit

\end{lstlisting}
Hint: Use \li{input_string = raw_input("Enter a string>>> ")} for user input.

\begin{warn}
In the end, my variable \li{function_pattern_string} was \emph{215 characters long}. You WILL make a mistake while defining \li{function_pattern_string}; do you want to try to debug a 215-character regular expression? Do NOT try to define it all at once!

Instead, use your previously defined regular expressions to make this easier. For example, either of the two following idioms will work:
\begin{lstlisting}
>>> key_1 = "basic"
>>> print ("This is a " + key_1 + " way to concatenate strings.")
This is a basic way to concatenate strings.
>>> format_dict = {"key_1": "basic", "key_2": "much more", "key_3": "advanced"}
>>> print ("This is a {key_2} {key_3} way to concatenate strings. It's {key_2} flexible.".format(**format_dict))
This is a much more advanced way to concatenate strings. It's much more flexible.
\end{lstlisting}
Keep in mind that you'll have to remove the line anchors from your previously defined regular expressions.

For reference, I used about ten lines to define \li{function_pattern_string} and used statements of the form \li{intermediate_pattern_string = r"(my regular expression here)".format()} four times.
\end{warn}

\label{prob:match_function_definition}
\end{problem}
