\lab{Algorithms}{Eigenvalue Solvers}{Eigenvalue Solvers}
\label{Ch:EigSolve}

\objective{Understand how eigenvalue solvers work. Implement a simple eigenvalue solver.}

Consider the following equation, where $A$ is a square matrix and $x$ is a non-zero vector and $\lambda$ is a scalar.

\begin{equation}
\label{eqn:eigval}
A x = \lambda x
\end{equation}

How do we find the eigenvalues (values for $\lambda$)?

\begin{equation*}
\begin{split}
 A x &= \lambda I x \\
A x - \lambda I x &= 0 \\
(A - \lambda I)x &= 0
\end{split}
\end{equation*}

Since $x$ is assumed to be a non-zero vector, that means $A-\lambda I$ must not have an inverse.  If $A-\lambda I$ is invertible then, $x=0$ and $\lambda$ cannot be an eigenvalue of $A$.  Since we know that $(A - \lambda I)^{-1}$ does not exist, then $det(A-\lambda I) = 0$.  This determinant yields what is known as the characteristic polynomial of $A$.

\begin{equation*}
 p(\lambda) = 0
\end{equation*}

If $A$ is $n \times n$, the degree of $p(\lambda) \leq n$.  Finding the roots of the characteristic polynomials gives the eigenvalues of $A$.  For small sizes of $A$, finding the eigenvalues is not difficult.  However, as the size of $A$ increases, the associated characteristic polynomial becomes more complex and find the roots of this polynomial becomes difficult.  Abel's Theorem  outlines the problem.

\begin{theorem}
\label{Theorem:Abel}
{\bf Abel's Impossibility Theorem:} There is no general algebraic solution for solving a polynomial equation of degree $n>4$.
\end{theorem}

Accordingly, this means that there is no method that will exactly find the eigenvalues of an arbitrary matrix. This is a significant result, and it means that we have to rely upon approximate methods: methods that converge to the eigenvalues. There are many such methods, but we will explore one of the simplest, the QR algorithm.

We illustrate the algorithm with an example. We choose a random vector $x_0$, which we can write as a linear combination of eigenvectors, $v_i$, of $A$.
\begin{equation}
\label{eqn:xnot}
x_0 = \sum v_i
\end{equation}

Using equation \ref{eqn:eigval}, we can rewrite \ref{eqn:xnot}.

\begin{equation*}
Ax_0 = \sum \lambda_i v_i
\end{equation*}

Now suppose that we examine $A^n x_0$. We can write the following:
\begin{equation*}
A^n x_0 = \sum \lambda_i^n v_i
\end{equation*}

What happens as $n \rightarrow \infty$? It turns out that we end up converging to the eigenvector that corresponds to the largest eigenvalue of A. Why does this happen?

The only assumption that the QR algorithm makes is that $x_0$ must be a linear combination of all of the eigenvectors (with non-zero coefficients). A random vector will always satisfy this condition, so we generally do not have to worry about choosing $x_0$.

This example algorithm only finds one eigenvector and eigenvalue. However, by combining this algorithm with the Gram-Schmidt algorithm we can find all of the eigenvectors and eigenvalues of any matrix. This is what the QR algorithm does.

\section*{QR Algorithm}

The following recurrence describes the QR Algorithm:

\begin{equation*}
A_0 = A, \quad A_k = Q_k R_k, \quad A_{k+1} = R_k Q_k
\end{equation*}

Where $Q_k R_k$ is the QR decomposition of $A_k$. 
Under very general conditions, $A_n$ will converge to a matrix of the form
\begin{equation*}
     \begin{pmatrix}
          B_1 &* & \cdots & * \\
           0     &B_2  &  \ddots & \vdots \\
           \vdots  & \ddots & \ddots & *  \\
           0 & \cdots & 0 & B_m
    \end{pmatrix}
\end{equation*}
where $B_i$ is either a $1 \times 1$ or a $2 \times 2$ matrix, and the eigenvalues of $A$ are the eigenvalues of the $B_i$. In the case that $A$ is symmetric, $A_k$ will be symmetric for all $k$, and $A_n$ will converge to a diagonal matrix.

\begin{problem}
Prove that $A_{k+1}$ is similar to $A_k$. This is central to the validity of the QR algorithm.
\end{problem}

\begin{problem}
Code the QR algorithm as written above. Have your function accept a matrix, $A$, and a number of iterations. Your function should return all the eigenvalues of $A$. You may use your QR decomposition function from Lab \ref{lab:QRdecomp}, or you can use the builtin version. Test your implementation with random symmetric matrices and compare your output to the output of \li{eig}. How many iterations are necessary? How large can $A$ be?
\end{problem}

\begin{problem}
Investigate your algorithm's performance on arbitrary matrices. What do you notice? How does your output differ from the built-in eigenvalue solver? How is it the same?
\end{problem}

\begin{problem}
Have your function additionally output the eigenvector corresponding to each eigenvalue. Hint: you have already made all the necessary calculations, you just need to store the information correctly.
\end{problem}

The QR algorithm is not the only iterative method used to find eigenvalues. For large sparse matrices MATLAB uses the Arnoldi iteration, which utilizes similar ideas as the QR algorithm while exploiting properties of the sparsity. Other methods include the Jacobi method and the Rayleigh quotient method.

It is important to remember that eigenvalue solvers can be wrong, particularly for matrices that are ill-conditioned. 
\begin{matlab}
For example using the MATLAB built in solver for the matrix generated by {\tt gallery(5)} yields the following output:
\begin{lstlisting}[style=matlab]
>> eig(gallery(5))
ans =
  -0.0405          
  -0.0118 + 0.0383i
  -0.0118 - 0.0383i
   0.0320 + 0.0228i
   0.0320 - 0.0228i
\end{lstlisting}

However, all of the eigenvalues of {\tt gallery(5)} are actually zero. This highlights the need to be conscious of the fact that eigenvalue solvers are really just doing an approximation.
\end{matlab}
