\lab{Algorithms}{Krylov Subspaces}{Finding Eigenvalues Using Iterative Methods}
\label{lab:kry_arnoldi}

\objective{Discuss simple Krylov Subspace Methods for finding eigenvalues and show some interesting applications.}

Krylov Subspace Methods are widely considered some of the most succesful numerical methods ever invented.
They are simple and robust iterative methods that can be used to find approximate solutions to linear systems and eigenvalue problems involving extremely large matrices.
One of the things that makes these numerical methods so succesful is that they do not require copies or modifications of the original matrix.
Krylov subspace methods are used to estimate some properties of a matrix based on how it acts on vectors through matrix multiplication.
This is especially useful for matrices that have symmetries that reduce storage and allow for faster matrix multiplication.
The general approach of Krylov subspace methods is to consider how a given matrix $A$ acts on the space spanned by $\lbrace x, Ax, A^2 x, ...A^N x \rbrace$ where $N$ is significantly larger than the number of rows of $A$.
The formation of these projections is usually based on either the Arnoldi Iteration or the Lanczos Iteration.
We will discuss both of these algorithms here.

\section*{The Arnoldi Iteration}

In Lab \ref{lab:Canonical_Transformations} we discussed how orthogonal transformations can be used to transform a matrix to Upper Hessenberg form.
We were able to find the eigenvalues of such Upper Hessenberg matrices in Lab \ref{lab:EigSolve}.
A similar approach will be used here.
The idea is to construct a set of vectors that allow us to roughly approximate the action of a very large matrix on the subspace.
This involves forming only a portion of the actual Hessenberg reduction of the Hessenberg factorization of the matrix.
You may recall from lab \ref{lab:QRdecomp} that the Modified Graham-Schmidt algorithm allows us to find an increasing number of orthogonal vectors but does not require that we run the algorithm to is completion to find a full set of them.
This process of projecting onto successively larger subspaces is called \textbf{The Arnoldi Iteration}.
The Arnoldi Iteration is outlined in Algorithm \ref{alg:arnoldi_iteration}.
We will now consider the ideas underlying the Arnoldi Iteration in further detail.

Let $A$ be an extremely large sparse matrix that we do not wish to modify.
We will assume that we have some sort of method for computing $A v$ for any vector $v$, but we will not make any other assumptions relating to how $A$ is stored or what properties it has.
Recall from Lab \ref{lab:Canonical_Transformations} that every matrix is orthogonally similar to an Upper Hessenberg matrix, so for matrices $Q$ and $H$ we have that $Q^* A Q = H$.

We would like to compute the columns of $Q$ one-by-one.
Let $q_k$ be the $k$'th column of $Q$ and $h_{i,j}$ be the $i,j$'th element of $H$.
We can 
From matrix multiplication we have that 
\[A q_n = h_{0, n} q_0 + \dots + h_{n, n} q_n + h_{n+1, n} q_{n+1}\]
This can be rewritten as
\[h_{n+1, n} q_{n+1} = A q_n - h_{0,n} q_0 - \dots - h_{n,n} q_n\]
Since each column of $Q$ has norm $1$, this allows computation of each $q_k$ based on each of the previous columns.
This recurrence relation does not provide any constraint on the first column of $Q$.
In practise, if we are searching for eigenvalues we can use a normalized random vector as the first column of $Q$.
These columns of $Q$ allow compution of partial reductions of $A$ to Hessenberg form without any modification of $A$.
This can seen as follows:

Let $Q_k$ be the first $k$ columns of $Q$.
Let $Q_u$ be the remaining columns of $Q$.
We then have that
\[H = Q^T A Q =
\begin{bmatrix}
Q_k^* A Q_k & Q_k^* A Q_u \\
Q_u^* A Q_k & Q_u^* A Q_u
\end{bmatrix}\]
Since $H$ is Upper Hessenberg, so is the matrix $Q_k^* A Q_k$ in the upper left corner.
Let $H_k = Q_k^* A Q_k$.

The recurrence relation for computing $q_{n+1}$ is the same expression that would be used to perform Graham Schmidt orthogonalization to project $A q_n$ orthogonal to the vectors $\lbrace q_0, \dots, q_n \rbrace$.
This allows for an alternate characterization of the Arnoldi iteration.
We first choose a normalized random vector $q_0$, then perform the Modified Graham Schmidt algorithm on the vectors $A q_0, A q_1, A q_2, \dots$.

\begin{algorithm}
\begin{algorithmic}[1]
\Procedure{arnoldi}{$A, b, Amul, k, tol=1E-8$}
	\State $m \gets A.shape[0]$					\Comment{Some initialization steps}
	\State $Q \gets \text{empty}\left(m, k+1\right)$
	\State $H \gets \text{zeros}\left( k+1, k\right)$
	\State $q_0 = b$							\Comment{Set $q_0$ equal to $b$.}
	\State $q_0 /= \|q_0\|_2$						\Comment{Normalize $q_0$.}
	\For{$j=0$, $j<k$}							\Comment{Perform the actual iteration}
		\State $q_{j+1} = Amul \left(q_j\right)$		\Comment{Compute $A q_j$}
		\For{$i=0$, $i<j+1$}					\Comment{Modified Graham Schmidt}
			\State $h_{i,j} = q \cdot q_{j+1}$		\Comment{Set values of $H$}
			\State $q_{j+1} -= h_{i,j} q_i$
		\EndFor
		\If{$|h_{j+1,j}|<tol$}					\Comment{Stop if $\|q_{j+1}\|_2$ is too small.}
			\State \pseudoli{return} $H[:j+1,:j]$, $Q[:,j]$
		\EndIf
		\State $h_{j+1,j} = \|z\|_2$				\Comment{Set subdiagonal element of $H$.}
		\State $q_{j+1} /= h_{j+1,j}$				\Comment{Normalize $q_{j+1}$}
	\EndFor
	\State \pseudoli{return} $H$, $Q$
\EndProcedure
\end{algorithmic}
\caption{The Arnoldi Iteration}
\label{alg:arnoldi_iteration}
\end{algorithm}

Notice that in Algorithm \ref{alg:arnoldi_iteration}, $k$ is the number of times to multiply by $A$.
This will result in a dimension $k+1$ Krylov Subspace.

\begin{problem}
Write a Python function that performs the Arnoldi Iteration given a matrix $A$, a normalized starting vector $b$, a function to multiply a vector on the left by $A$, and a number $n$ of steps to perform.
Also have it accept a tolerance parameter that defaults to \li{1E-8}.
Have it return the computed $H_n$ and $Q_n$.
\end{problem}

% Make a 3D plot showing the projection. Label the initial vector x and Ax.

\section*{Finding Eigenvalues Using Arnoldi Iteration}

% Plot how ritz values converge to eigenvalues

% Have them code it. Comment out an option to let them use their own solver. Uncomment that portion once the eigenvalue lab has been reworked.

\section*{Lanczos Iteration}

% Plot some pseudospectra?

% Have them code a version with a while loop that goes until convergence is reached. Use it to find the norm of random matrices.
% Point toward use of largest singular values in PCA and LSI in volume 3.

\section*{Polynomial root finding}

% Perform Arnoldi iteration on companion matrix. Give them a routine that computes Ax without explicitly constructing it.

% Plot basins of convergence for poynomial root finding.

\section*{Arnoldi Iteration in SciPy}

% Use it to find the Fiedler value to determine whether or not a graph is connected.


% Show Ghost Eigenvalues?

% Add another application if there's space.

% Eigvals of DFT computed via the FFT? This one is good because it uses the Lanczos iteration.

% Applications of eigenvalues of discretized representations of differential operators?