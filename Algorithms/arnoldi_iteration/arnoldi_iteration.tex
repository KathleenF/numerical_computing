\lab{Algorithms}{Krylov Subspaces}{Finding Eigenvalues Using Iterative Methods}
\label{lab:kry_arnoldi}

\objective{Discuss simple Krylov Subspace Methods for finding eigenvalues and show some interesting applications.}

Krylov Subspace Methods are widely considered some of the most succesful numerical methods ever invented.
They are simple and robust iterative methods that can be used to find approximate solutions to linear systems and eigenvalue problems involving extremely large matrices.
One of the things that makes these numerical methods so succesful is that they do not require copies or modifications of the original matrix.
Krylov subspace methods are used to estimate some properties of a matrix based on how it acts on vectors through matrix multiplication.
This is especially useful for matrices that have symmetries that reduce storage and allow for faster matrix multiplication.
The general approach of Krylov subspace methods is to consider how a given matrix $A$ acts on the space spanned by $\lbrace x, Ax, A^2 x, ...A^N x \rbrace$ where $N$ is significantly larger than the number of rows of $A$.
The formation of these projections is usually based on either the Arnoldi Iteration or the Lanczos Iteration.
We will discuss both of these algorithms here.

\section*{The Arnoldi Iteration}

% Discuss how it works

% Have them code it as a problem. Do it taking an explicit value for the (maximum?) number of iterations to perform.

% Make a 3D plot showing the projection. Label the initial vector x and Ax.

\section*{Finding Eigenvalues Using Arnoldi Iteration}

% Plot how ritz values converge to eigenvalues

% Have them code it. Comment out an option to let them use their own solver. Uncomment that portion once the eigenvalue lab has been reworked.

\section*{Lanczos Iteration}

% Plot some pseudospectra?

% Have them code a version with a while loop that goes until convergence is reached. Use it to find the norm of random matrices.

\section*{Polynomial root finding}

% Perform Arnoldi iteration on companion matrix. Give them a routine that computes Ax without explicitly constructing it.

% Plot basins of convergence for poynomial root finding.

\section*{Arnoldi Iteration in SciPy}

% Use it to find the Fiedler value to determine whether or not a graph is connected.


% Show Ghost Eigenvalues?

% Add another application if there's space.

% Eigvals of DFT computed via the FFT? This one is good because it uses the Lanczos iteration.

% Applications of eigenvalues of discretized representations of differential operators?