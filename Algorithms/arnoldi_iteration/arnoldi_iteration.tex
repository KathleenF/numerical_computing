\lab{Algorithms}{Krylov Subspaces}{Finding Eigenvalues Using Iterative Methods}
\label{lab:kry_arnoldi}

\objective{Discuss simple Krylov Subspace Methods for finding eigenvalues and show some interesting applications.}

\section*{The Arnoldi Iteration}

% Discuss how it works

% Have them code it as a problem. Do it taking an explicit value for the (maximum?) number of iterations to perform.

% Make a 3D plot showing the projection. Label the initial vector x and Ax.

\section*{Finding Eigenvalues Using Arnoldi Iteration}

% Plot how ritz values converge to eigenvalues

% Have them code it. Comment out an option to let them use their own solver. Uncomment that portion once the eigenvalue lab has been reworked.

\section*{Lanczos Iteration}

% Plot some pseudospectra?

% Have them code a version with a while loop that goes until convergence is reached. Use it to find the norm of random matrices.

\section*{Polynomial root finding}

% Perform Arnoldi iteration on companion matrix. Give them a routine that computes Ax without explicitly constructing it.

% Plot basins of convergence for poynomial root finding.


% Show Ghost Eigenvalues?

% Add another application if there's space.

% Eigvals of DFT computed via the FFT? This one is good because it uses the Lanczos iteration.

% Applications of eigenvalues of discretized representations of differential operators?