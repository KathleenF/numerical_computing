\lab{Algorithm}{Change of Basis}{Change of Basis}
\label{lab:ChangeBasis}

\objective{Understand how to change the basis of a set of points.}

\section*{Basis}

A basis for a vector space is a set of vectors such that every vector in the space can be expressed uniquely as a linear combination of the basis vectors. In this lab we will take the coordinates in 2-d space and do various affine transformations. For all these exercises we will use the points
\begin{lstlisting}
x=[-1.5,-1.,-.5,0.,.5,1.,1.5,.75,-.75]
y=[0.,-1.,-2.,-2.,-2.,-1.,0.,2.,2.]
\end{lstlisting}
Let be the matrix where each column in a point and M is the matrix that will change the basis. So $M*A$ will be the set of points in the new basis. In our cases the first row of $A$ is $x$ and the second row, $y$. 

\section*{Strench}
To strech a set of points M will be a diagonal matrix where the value in each position is the streach in that direction

\begin{problem}
Write a function that will accepts a matrix of points and how much to strench them in each direction. Have the function plot the transformed points.
\end{problem}

\section*{Rotation}
To do a rotation clockwise of angle $\theta$ let
\[
M = \begin{pmatrix}
\cos(\theta) & -\sin(\theta) \\
\sin(\theta) & \cos(\theta) 
\end{pmatrix}
\]


\begin{problem}
Write a function that will accepts a matrix of points and how many radians to rotate the points. Have the function plot the transformed points.
\end{problem}

\section*{Shift}
In order to shift a set of points you add to the coordinate how much you would shift it in that direction. You can use array brodcasting to do this.

\begin{problem}
Write a function that will accepts a matrix of points and how much to shift them in each direction. Have the function plot the transformed points.
\end{problem}

\section*{Combination}
Say you have points in a rotated basis and you want to stretch them along that basis. You left mutiply the strech matrix $S$ by the rotation matrix $R$. So $R*S*A$ will strench the points in the rotation in R. 

\begin{problem}
Write a function that will accepts a matrix of points and strench, rotate and shift them.
\end{problem}


\section*{Images}

An Image is a 3d array where the first two demensions are the location and the 3rd demension is the RGB content. To apply the above tranformations one would need to moved the position of the RGB arrays to match where the new transformation would have them. Strenching is done by
 interpolation.
\begin{problem}
Write a function that will accepts an image and how many radians to rotate the image. The function will rotate the image around the center of the image and then show the rotated image. HINTS: Make an array of ones that is $1.5$ times greater than the max of the hieght and width of the original image. Change the x,y cordinates in the original image so the the center is $(0,0)$. Rotate the image and then center the image. You might need to use for loops. Allow your test cases to be small images.
\end{problem}


% \begin{thebibliography}{99}
% \bibitem{brualdi09}
% Richard Brualdi, 
% \emph{Introductory Combinatorics}.
% Pearson, New Jersey, 5th edition, 2009.
% 
% \end{thebibliography}
