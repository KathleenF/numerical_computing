\lab{Algoirthms}{Database Essentials}{Database Essentials}
\objective{Familiarization with MySQL prompt and basic queries
\label{lab:DbEssentials}

\section*{Introduction to MySQL}

Databases are an essential part of computation.  We will learn how to use MySQL databases in this section.  MySQL is open-source software that is easy to use and free.  It also works well with Python.  All of this makes it a good choice for our applications.  Other options exist, but most of what we learn here will translate easily to other platforms.

We do not explain installation and setup of MySQL here.  See an appendix (TODO:  Make an appendix with a tutorial).

MySQL works by running a server that keeps track of databases and listens for queries.  There are several ways to interact with the server.  In this section, we will be using the MySQL terminal launched from the command line.  Let's begin by starting the server.  In your command prompt, navigate to the location of your MySQL installation. Now, type

\begin{lstlisting}[style=ShellInput]
/bin/mysqld
\end{lstlisting}

Now that we are sure that the server is on, let's start the MySQL terminal so that we can talk to it.  From the command prompt, type

\begin{lstlisting}[style=ShellInput]
mysql -u root -p
\end{lstlisting}

You should get a new prompt that looks like this

\begin{lstlisting}
Welcome to the MySQL monitor.  Commands end with ; or \g.
Your MySQL connection id is 1
Server version: 5.5.18 MySQL Community Server (GPL)

Copyright (c) 2000, 2011, Oracle and/or its affiliates. All rights reserved.

Oracle is a registered trademark of Oracle Corporation and/or its
affiliates. Other names may be trademarks of their respective
owners.

Type 'help;' or '\h' for help. Type '\c' to clear the current input statement.

mysql> 
\end{lstlisting}

You are now ready to start querying the server.

\section*{Basic Commands in MySQL}

\subsection{Creating Databases and Tables}

MySQL can store muliple databases, and each database will have several tables.  To see the databases in MySQL, we use the {\tt SHOW DATABASES;} command.

\begin{lstlisting}

mysql> SHOW DATABASES;
+--------------------+
| Database           |
+--------------------+
| information_schema |
| mysql              |
| performance_schema |
+--------------------+
3 rows in set (0.03 sec)

\end{lstlisting}

Note that the command ends with a semicolon. Every command we will see will also end with the same character.  Your terminal may show additional databases to these.  Each of these three databases store information about your installation of MySQL and the databases that are stored on the server.  They have their uses, but for now we will ignore them.

Now we will go over some basic queries that you can execute from the command line.  We want to be able to get data from MySQL server, but we don't have a database to get it from yet.  We can use the {\tt CREATE} command to create a new database.  In this example, we will create a student database with three tables.

\begin{lstlisting}
mysql>CREATE DATABASE students;
Query OK, 1 row affected (0.00 sec)

mysql> show databases;
+--------------------+
| Database           |
+--------------------+
| information_schema |
| mysql              |
| performance_schema |
| students           |
+--------------------+
4 rows in set (0.00 sec)

\end{lstlisting}

Now we add some tables to the students database.  First, we need to tell MySQL which database we want to work on with the {\tt USE} command.

\begin{lstlisting}

mysql> USE students;
Database changed

\end{lstlisting}

Now we use the {\tt CREATE} command to create a table and specify the column names of the table.

\begin{lstlisting}

mysql> CREATE TABLE student_information (StudentID INT NOT NULL, Name VARCHAR(20), SocSecurity INT, MajorCode INT);
Query OK, 0 rows affected (0.68 sec)

\end{lstlisting}

This creates a student information table with coloumns for each individuals student id, social security number, major, and name.  The arguments in parentheses are the column names followed by the datatype that entries in that column will take.  The {\tt INT} datatype allows up to 32-bit integers and {\tt VARCHAR(20)} allows strings of variable length up to 20 characters.  There are dozens of options to choose from.  The online MySQL documentation is a good resource to explore them.

Now we add another table that matches students to courses that they have taken and their grades.

\begin{lstlisting}

mysql> CREATE TABLE student_classes (StudentID INT NOT NULL, CourseID INT, Grade VARCHAR(2));
Query OK, 0 rows affected (0.26 sec)

\end{lstlisting}

\begin{exercise}

In this exercise you will create two new tables using the {\tt CREATE TABLE} syntax.  The first table will be called MajorInfo and have a columns called MajorCode and MajorName.  MajorCode should have the INT datatype and MajorName should be VARCHAR(20).

The second table will be called CourseInfo and have columns called CourseID and CourseName, also INT and VARCHAR(20) respectively.

\end{exercise}

\subsection{Inserting Data}

In this section we will insert data into the tables that we have created.  First we will show how to manually add data using the {\tt INSERT} command.  {\tt INSERT} requires that we specify which table that we wish to modify and the values for each column.  For example

\begin{lstlisting}
mysql> INSERT INTO student_information VALUES(55, "John Smith", 372897382, 2);
Query OK, 1 row affected (0.20 sec)
\end{lstlisting}

We may also remove the line we just inserted using the {\tt DELETE} command.

\begin{lstlisting}

mysql> DELETE FROM student_information WHERE StudentID=55;
Query OK, 1 row affected (0.16 sec)

\end{lstlisting}

These commands are useful for table maintenance or where it may be automated, but MySQL is a very powerful tool for dealing with large amounts of data.  Manually inserting and deleting in such a table is ornerous in even simple cases and essentially impossible for a moderately sized table.  If we have some data in a file on our computer, we may use the {\tt LOAD} command to automatically move it into our database.

\begin{lstlisting}

mysql> LOAD DATA LOCAL INFILE "./students.dat" INTO TABLE student_information FIELDS TERMINATED BY ",";
Query OK, 10 rows affected (0.07 sec)

\end{lstlisting}

{\tt LOCAL INFILE} specifies that we are loading a file from a location on our local machine.  We then specify the path to the file and which table to load the data into.  Finally, we specify what character separates the columns.  Even for large files this operation moves very fast.

\begin{exercise}

Download students.dat, class_info.dat, classes.dat, and major_info.dat and use the {\tt LOAD} command to insert the data in each into student_information, ClassInfo, student_classes and MajorInfo respectively.

\end{exercise}

\begin{exercise}

Using python, write a program that will accepts a number of rows and returns a 10 column matrix with the specified number of rows and write it to a file with each column separated by commas.  The first column should be in ascending numeric order, but the others should have random numbers in them.  Generate matrices with 100, 1000, 10000, and 100000 rows.  Load each into a different table in a new database and time the load operation.  How does the time increase?

\end{exercise}

\section{Querying and Joining Tables}

Now that we have loaded data in the form of our student database we will learn to query and join the tables.  The first command that we will use is {\tt SELECT}, which returns specified rows of a table.  For example

\begin{lstlisting}
mysql> SELECT Name FROM student_information;
+---------------------+
| Name                |
+---------------------+
|  Jared Webb         |
|  Alexander Zaitzeff |
|  Rachel Suggs       |
|  Jeff Humphreys     |
|  Abe Frandsen       |
|  Tyler Jarvis       |
|  Jessica Purcell    |
|  Janice Joplin      |
|  John Lennon        |
|  Tupac Shakur       |
+---------------------+
10 rows in set (0.00 sec)
\end{lstlisting}

After {\tt SELECT} we specify a column and a table to query, and MySQL returns the requested rows.  We may also add conditions to our command to get more refined results.

\begin{lstlisting}

mysql> SELECT Name FROM student_information WHERE StudentID = 4;
+-----------------+
| Name            |
+-----------------+
|  Jeff Humphreys |
+-----------------+
1 row in set (0.00 sec)

\end{lstlisting}

Or we may select more than one column

\begin{lstlisting}

mysql> SELECT Name, SocSecurity FROM student_information WHERE StudentID = 4;
+-----------------+-------------+
| Name            | SocSecurity |
+-----------------+-------------+
|  Jeff Humphreys |   736452198 |
+-----------------+-------------+
1 row in set (0.00 sec)

mysql> SELECT * FROM student_information WHERE StudentID = 4;
+-----------+-----------------+-------------+-----------+
| StudentID | Name            | SocSecurity | MajorCode |
+-----------+-----------------+-------------+-----------+
|         4 |  Jeff Humphreys |   736452198 |         3 |
+-----------+-----------------+-------------+-----------+
1 row in set (0.00 sec)

mysql> SELECT * FROM student_information;
+-----------+---------------------+-------------+-----------+
| StudentID | Name                | SocSecurity | MajorCode |
+-----------+---------------------+-------------+-----------+
|         1 |  Jared Webb         |   123456789 |         1 |
|         2 |  Alexander Zaitzeff |   987654321 |         1 |
|         3 |  Rachel Suggs       |   431256789 |         2 |
|         4 |  Jeff Humphreys     |   736452198 |         3 |
|         5 |  Abe Frandsen       |   172645382 |         4 |
|         6 |  Tyler Jarvis       |   174382645 |         3 |
|         7 |  Jessica Purcell    |   827635142 |         2 |
|         8 |  Janice Joplic      |   987263512 |         1 |
|         9 |  John Lennon        |   192837641 |         3 |
|        10 |  Tupac Shakur       |   192837412 |         2 |
+-----------+---------------------+-------------+-----------+
10 rows in set (0.00 sec)

\end{lstlisting}

We may join tables by columns using the {\tt INNER JOIN} command.  This is a very powerful tool for uniting data across many tables.  In MySQL we can use {\tt INNER JOIN} in conjunction with {\tt SELECT} to query data across many tables into place.  For example we can inner join student_classes and student_information on the StudentID column to display student information and grades in one table.

\begin{lstlisting}

mysql> SELECT * FROM student_information INNER JOIN student_classes ON student_information.StudentID = student_classes.StudentID;
+-----------+---------------------+-------------+-----------+-----------+---------+-------+
| StudentID | Name                | SocSecurity | MajorCode | StudentID | ClassID | Grade |
+-----------+---------------------+-------------+-----------+-----------+---------+-------+
|         1 |  Jared Webb         |   123456789 |         1 |         1 |       4 |  C    |
|         1 |  Jared Webb         |   123456789 |         1 |         1 |       3 |  B    |
|         2 |  Alexander Zaitzeff |   987654321 |         1 |         2 |       4 |  A    |
|         2 |  Alexander Zaitzeff |   987654321 |         1 |         2 |       3 |  A    |
|         3 |  Rachel Suggs       |   431256789 |         2 |         3 |       2 |  C    |
|         2 |  Alexander Zaitzeff |   987654321 |         1 |         2 |       1 |  B    |
|         4 |  Jeff Humphreys     |   736452198 |         3 |         4 |       1 |  A    |
|         5 |  Abe Frandsen       |   172645382 |         4 |         5 |       2 |  C    |
|         5 |  Abe Frandsen       |   172645382 |         4 |         5 |       3 |  C    |
|         6 |  Tyler Jarvis       |   174382645 |         3 |         6 |       4 |  D    |
|         6 |  Tyler Jarvis       |   174382645 |         3 |         6 |       2 |  A    |
|         6 |  Tyler Jarvis       |   174382645 |         3 |         6 |       1 |  B    |
|         7 |  Jessica Purcell    |   827635142 |         2 |         7 |       2 |  A    |
|         7 |  Jessica Purcell    |   827635142 |         2 |         7 |       1 |  C    |
|         8 |  Janice Joplic      |   987263512 |         1 |         8 |       2 |  D    |
|         8 |  Janice Joplic      |   987263512 |         1 |         8 |       1 |  A    |
|         5 |  Abe Frandsen       |   172645382 |         4 |         5 |       4 |  A    |
|         9 |  John Lennon        |   192837641 |         3 |         9 |       2 |  B    |
|         9 |  John Lennon        |   192837641 |         3 |         9 |       3 |  C    |
|        10 |  Tupac Shakur       |   192837412 |         2 |        10 |       1 |  A    |
|        10 |  Tupac Shakur       |   192837412 |         2 |        10 |       2 |  A    |
+-----------+---------------------+-------------+-----------+-----------+---------+-------+
21 rows in set (0.05 sec)

\end{lstlisting}

By specify columns, we can display the grades of one student.

\begin{lstlisting}

mysql> SELECT Name, Grade FROM student_information INNER JOIN student_classes ON student_information.StudentID = student_classes.StudentID WHERE student_information.StudentID=6;
+---------------+-------+
| Name          | Grade |
+---------------+-------+
|  Tyler Jarvis |  D    |
|  Tyler Jarvis |  A    |
|  Tyler Jarvis |  B    |
+---------------+-------+
3 rows in set (0.05 sec)

\end{lstlisting}

\begin{exercise}

Use {\tt INNER JOIN} and {\tt SELECT} to display the student_information table, but show the students majors and classes instead of ClassID and MajorID.

\end{exercise}

