\lab{Algorithms}{B-Splines}{B-Splines}
Though B\'{e}zier curves are good for a variety of things, they have certain limitations.
\begin{itemize}
\item As the number of control points increases it becomes very expensive to compute the points on the curve.
\item Changes in the placement of a single control point affect the shape of the entire curve.
\item Since a change in any control point affects the entire curve, it is necessary to recompute the entire curve to account for a change in a single control point.
\item As the number of points increases, individual points have progressively less affect on the portions of the curve that lie nearest to them.
\end{itemize}

B-Splines are an ideal way to answer these limitations.
A B-Spline is, roughly speaking, a piecewise B\'{e}zier curve.

\section*{B-Spline Basis Functions}

In the previous lab we introduced a way to represent B\'{e}zier curves as a linear combination of Bernstein Polynomials.
Notice the convenient property that the coefficient of each Bernstein polynomial for the B\'{e}zier curve formed from a given control point is simply the control point itself.

This is a useful property.
B-splines are a generalization of B\'{e}zier curves which allow us to use piecewise basis functions.
Using piecewise basis functions allows us to make local changes to the curve we are representing.
We can also choose basis functions that are zero on significant portions of the domain so that we do not have to recompute the entire curve when we change only one control point.
Choosing basis functions that are 0 nearly everywhere also makes control of the curve easier because changes from each control point are only applied locally.
Another useful effect is that this makes the individual pieces of the B-Spline more responsive to changes in the control points that are used in creating it.
This, again, makes the curve easier to control.
The idea is that each point only has a local influence on the curve, so each point can have a larger effect on a smaller portion of the curve.

We would like to be able to do this without loosing the useful properties of B\'{e}zier curves.
Using basis functions of any sort guarantees that each point of the curve will be a linear combination of the control points.
It would be best if we could make it so that this curve has the convex hull property (i.e. that it lies within the area bounded by the outermost control points).
B\'{e}zier curves also allow easy computation of their derivatives, so we would hope that B-Splines would allow this as well.
One of the largest constraints we need is that the curve we are forming be continuous with a given number of continuous derivatives.
The construction of the B-Spline Basis functions takes all of these factors into account.

There are a number of ways the B-Spline basis functions can be defined.
A common approach is to use recursion.
Let $t = \lbrace t_0, t_1, ... , t_m \rbrace$ be a nondecreasing sequence of real numbers.
$t$ is called the knot vector.
The $t_i$ are called the knots.
Note that each $t_i$ is not necessarily distinct.
$N_{i,k}(x)$, the $i$'th B-Spline basis function of degree $k$ is defined as:

\begin{equation}
N_{i,0}(x) =
\begin{cases}
1 & \text{if } t_i \leq x < t_{i+1} \\
0 & \text{otherwise}
\end{cases}
\end{equation}

\begin{equation}
N_{i,p}(x) = \frac{x - t_i}{t_{i+k} - t_i} N_{i,k-1}(x) + \frac{t_{i + k + 1} - x}{t_{i + k + 1} - t_{i + 1}} N_{i+1,k-1}(x)
\end{equation}

This is known as the Cox-De Boor recursion formula.
This algorithm is the De Boor algorithm.
When implemented properly, this algorithm is both fast and numerically stable.

Notice that we defined $t$ to be nondecreasing, which means some $t_i$ may be repeated.
When programming this algorithm as it is currently written, be careful to avoid division by zero.
When we get the indeterminate form $\frac{0}{0}$ we define these terms to be zero.

Also note that $N_{i,0}$ is a step function which is zero except on $[t_i, t_{i+1})$.
The other basis functions are piecewise polynomials of degree $k$ that are only nonzero on $[t_{i-k}, t_{i+1+k})$.
This is also nice because computation of a set of basis functions requires only a knot vector $t$ and a degree $k$, so we can predefine basis functions before we know the actual positions of the control points.

\begin{warn}
When performing the computation of each of the terms in the De Boor Algorithm, you should very careful about division by $0$.
Whenever division by $0$ occurs in either term, you should replace the value for \emph{that specific term} by $0$.
This should be done for each term independently so that, if you are using NumPy's float types, a value of \li{inf} or \li{nan} does not propagate through the recursion in the algorithm.
\end{warn}

\begin{problem}
Use the De Boor algorithm to write a recursive python function to evaluate a b-spline basis function for some $u$ between the maximum and minimum of a given knot vector $t$.
\end{problem}

Upon considering the computation involved in the previous problem, we see that this recursion, particularly for higher order splines, involves a massive amount of repetitive calculation.
Many of the performance costs can be avoided by rewriting the algorithm using explicit loops and avoiding unnecessary computations.

\begin{comment}
% These comments and this problem could be one way to expand this lab later on.
% Depending on how long the lab is, we will probably want to give them pseudocode for this version of the algorithm.

First, notice that for splines of order $2$ and higher, we actually compute the values of some splines multiple times.
A simple way to avoid this is to figure out which of the 0-order splines we will actually use in our computation, compute them all, then compute all the needed splines of order 1, 2, etc.

There is also some redundant computation in the computation of the coefficients used at each stage of the recursion.
This can be eliminated by using good control structure and a temporary variable.

We will label the left and right coefficients in the formula $L$ and $R$ respectively, so we have $L(i, k, x) = \frac{x - t_i}{t_{i + k} - t_i}$ and $R(i, p, u) = \frac{t_{i + k + 1} - x}{t_{i + k + 1} - t_{i + 1}}$.
Notice that $L(i + 1, k, x) = 1 - R(i, k, x)$.
We can eliminate much of the duplicate computation by computing the new left hand side coefficient the iteration before we actually need it.
This avoids nearly all the repeated computation.

\begin{problem}
Write a function that uses loops instead of recursion to compute the values for all the b-spline basis functions of a given power $k$ for a given array of $t$ values.
Return the answer as a two dimensional array with the results for each polynomial stored in each of the rows of the array.
\end{problem}

It is worth noting that you can remove further excess computation when evaluating a single function and even further when evaluating a single function at a single point.
This can be done by figuring out in advance which of the $N_{i,k}$ will be nonzero and only iterating over those terms.
This approach may or may not be faster depending on the form of the problem.

\end{comment}

Scipy has some built in functions and a built in class for B-splines.
They are all part of the scipy.interpolate package.

As of SciPy 14, \li{scipy.interpolate} also includes the \li{BPoly} class that allows for easy computation of the basis functions for a B-spline.

Several of the functions built in to SciPy are wrappers around the Fortran package FITPACK.
One such example is the function \li{scipy.interpolate.splev}.
This function is used to evaluate a spline with knot vector $t$, basis function coefficients $c$, and degree $k$ at a set of points $x$.
The calling convention for this function is \li{splev(x, (t, c, k))}.
If $c$ is a multi-dimensional array, the function is applied to each row of $c$.
This is equivalent to using each column of $c$ as a control point in a higher dimension.

The package \li{scipy.interpolate} also includes several routines designed for easy interpolation with b-splines.
It also has routines designed for integration and differentiation of B-splines.

\begin{problem}
Use \li{scipy.integrate.splev} to generate a plot of a b-spline with randomly chosen control points in $\mathbb{R}^2$.
Generate the same plot with your own implementation of the De Boor's algorithm.

Use the following knot vector (where $n$ some integer and $k$ is the degree of the desired spline):
\begin{lstlisting}
t = np.array([0]*(k) + range(n) + [n]*(k+1))
\end{lstlisting}
This knot vector will give you $k + n$ different nonzero basis functions on the interval $\left[0, n\right]$.
\end{problem}
