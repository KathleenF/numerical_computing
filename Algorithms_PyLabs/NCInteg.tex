\lab{Algorithms}{Newton-Cotes Integration}{Newton-Cotes Integration}

\objective{Explain Newton-Cotes Integration}

{\bf Outline:}
\begin{itemize}
\item Intro: We need a simple, higher-order method for approximating integrals. Here we assume evenly spaced points.
\item The simple approach: naive quadrature, midpoint rule.
\item Higher-order Approximation: Use the Lagrange interpolation
\item This yields weights that are fixed for evenly-spaced evaluation points, that are still higher-order approximations
\item Problem: At very high orders, with evenly-spaced points, we get Runge's phenomenon.
\end{itemize}

\begin{problem}
Derive Simpson's Rule. Code and use. Note exactness for lower order polynomials.
\end{problem}

\begin{problem}
Develop Higher Order generator. Simpson's 3-8 Rule
\end{problem}

\begin{problem}
Demonstrate Runge's phenomenon. Explain how to use a composite rule to overcome this.
\end{problem}
