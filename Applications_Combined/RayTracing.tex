\lab{Applications}{Ray Tracing}{Ray Tracing}

\objective{Understand and implement some ray tracing stuff}

{\bf Outline}

\begin{itemize}
\item Explain the problem of simulating light being cast from a source: most of it doesn't make it to the camera.
\item Explain the approach of raytracing: go backwards from the camera.
\item Different types of rays: reflection refraction and shadow.
\item Work an example of shadow maybe (sphere shaded partially by a cube)?
\item Explain computational burden (lots of rays). Tradeoff between realism and speed. Used more in applications that can take time (such as movies).
\item Explain refraction well (angles incident to surface etc.).
\end{itemize}

\begin{problem}
Model a movie screen with a glass shape (ellipse? Sphere?) between the viewer and the screen. What does the movie look like? This will combine refraction and raytracting.
\end{problem}

