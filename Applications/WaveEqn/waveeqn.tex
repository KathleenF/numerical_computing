\lab{Applications}{Wave Phenomena}{Wave Phenomena}
\label{lab:waveeqn}


\section{Advection Equation}
The advection or transport equation is given by $u_t + s u_x = 0$, where $s$ is a nonzero constant.  
Consider the Cauchy problem 
\begin{align*}
	& u_t + su_x = 0, \quad -\infty < x < \infty,\\\
	& u(x,0) = f(x).
\end{align*}
The function $f(x)$ may be thought of as an initial wave or signal. The general solution of this initial boundary value problem is $u(x,t) = f(x-st)$ (check this!). The solution $u(x,t)$ takes the signal $f(x)$ and moves it along at a constant speed $s$ - to the right if $s > 0$, and to the left if $s < 0$. 

\section{Wave Equation}
Many different wave phenomena can be described using a hyperbolic pde called the wave equation. 
These wave phenomena occur in fields such as electromagnetics, fluid dynamics, and acoustics. 
This equation is given by 
\begin{align}
	u_{tt} &= s^2 \triangle u.
\end{align}
This equation (in 1D) is commonly derived by describing the motion of a string vibrating in a plane. Another nice derivation for the 1D equation can also be done by using Hooke's law from the theory of elasticity (look at the Wikipedia article on the wave equation).

The wave equation is usually seen in the context of an initial boundary value problem. 
In 1 dimension, this can take the form 
\begin{align*}
	u_{tt} &= s^2 u_{xx}, \quad 0 < x < l, \quad t > 0\\
	u(0,t) &= u(l,t) = 0, \\
	u(x,0) &= f(x),\\ 
	u_t(x,0) &= g(x).
\end{align*}

\subsection{Motivation for the wave equation}
Consider the 1D wave equation 
\[u_{tt} = s^2 u_{xx}.\]
After making the change of variables $(\xi,\eta) = (x-st, x + st)$ and using the chain rule, we find that the equation $u_{tt} = s^2 u_{xx}$ is equivalent to $u_{\xi \eta} = 0$. The general solution of this last equation is 
\[ u(\xi, \eta) = F(\xi) + G(\eta) \]
for some scalar functions $F$ and $G$. In $(x,t)$ coordinates the solution is 
\[u(x,t) = F(x-st) + G(x+st).\]
Thus the general solution of the wave equation is the sum of two parts: one is a signal traveling to the right with constant speed $|s|$, and the other is a signal traveling to the left with speed $|s|$.

\subsection{Numerical solution of the wave equation}
We look to approximate $u(x,t)$ on a grid of points $(x_j,t_m)_{j=0,m=0}^{J,M}$.
Denote the approximation to $u(x_j,t_m)$ by $U_{j,m}$.
Recall that the centered approximations in space and time are
\begin{align*}
D_{tt} U_{j,m} = \frac{U_{j,m+1} -2 U_{j,m} + U_{j,m-1}}{(\triangle t)^2} ,\\
D_{xx} U_{j,m} = \frac{U_{j+1,m} -2 U_{j,m} + U_{j-1,m}}{(\triangle x)^2} .
\end{align*}
% \[ 
% u_{tt}(x_j,t_m) = \frac{u(x_j,t_m+k) -2 u(x_j,t_m) + u(x_j,t_m-k)}{k^2} + \mathcal{O}(k^2).
% \]
% 
% \[ 
% u_{xx}(x_j,t_m) = \frac{u(x_j+h,t_m) -2 u(x_j,t_m) + u(x_j-h,t_m)}{k^2} + \mathcal{O}(h^2).
% \]
The resulting method is given by 
\begin{align*}
	&\frac{U_{j,m+1} -2 U_{j,m} + U_{j,m-1}}{(\triangle t)^2} = s^2 \frac{U_{j+1,m} -2 U_{j,m} + U_{j-1,m}}{(\triangle x)^2}, \\
	% &U_{j,m+1} = 2 U_{j,m} - U_{j,m-1} + \lambda ^2 (U_{j+1,m} -2 U_{j,m} + U_{j-1,m}), \\
	&U_{j,m+1} =  - U_{j,m-1} + 2 (1-\lambda^2) U_{j,m} + \lambda ^2 (U_{j+1,m} + U_{j-1,m}),
\end{align*}
where $ \lambda  =  s(\triangle t)/(\triangle x)$. This method may be written in matrix form as 
\[U^{m+1} = AU^{m} - U^{m-1} \]
where 
\[
A = 
\left[\begin{array}{cccc}2(1-\lambda^2) & \lambda^2 &  &  \\ \lambda^2 & 2(1-\lambda^2) & \lambda^2 &  \\ \ddots & \ddots & \ddots &  \\ & \lambda^2 & 2(1-\lambda^2) & \lambda^2 \\  &  & \lambda^2 & 2(1-\lambda^2)\end{array}\right]
\]
and
\[ U^m = 
\left[\begin{array}{c}U_{1,m} \\U_{2,m} \\\vdots \\U_{J-1,m}\end{array}\right].
\]


% \[
% \left[\begin{array}{c}U_{1,j+1} \\U_{2,j+1} \\\vdots \\U_{m-1,j+1}\end{array}\right] = 
% \left[\begin{array}{cccc}2(1-\lambda^2) & \lambda^2 &  &  \\ \lambda^2 & 2(1-\lambda^2) & \lambda^2 &  \\ \ddots & \ddots & \ddots &  \\ & \lambda^2 & 2(1-\lambda^2) & \lambda^2 \\  &  & \lambda^2 & 2(1-\lambda^2)\end{array}\right]
% \left[\begin{array}{c}U_{1,j} \\U_{2,j} \\\vdots \\U_{m-1,j}\end{array}\right]  - 
% \left[\begin{array}{c}U_{1,j-1} \\U_{2,j-1} \\\vdots \\U_{m-1,j-1}\end{array}\right] 
% \]
In the matrix equation above, we have already used the boundary conditions to determine that 
$U_{0,m} = U_{J,m} = 0$ at each time $t_m$. 
Note that to obtain the approximation $U_{j,m+1}$ of $u(x_j,t_{m+1})$, the method uses the value of the approximation $U$ at the previous two time steps. We can find the solution for the first two time steps by using the initial conditions. Directly using the initial conditions gives an approximation at $t = t_0 = 0:$
\[
U_{j,0} = f(x_j), \quad 1 \leq j \leq J-1.
\]

To obtain an approximation at the second time step, we consider the Taylor expansion 
\begin{align*}
	u(x_j,t_1) &= u(x_j, 0) + u_t(x_j,0) \triangle t + u_{tt}(x_j,0) \frac{\triangle t^2}{2} + u_{ttt}(x_j,t_1^*) \frac{\triangle t^3}{6}.
\end{align*}
Recalling that the solution $u(x,t)$ satisfies the wave equation, we substitute in expressions from our initial conditions: 
\begin{align*}
	u(x_j,t_1) &= u(x_j, 0) +  g(x_j) \triangle t+ s^2 f''(x_j)\frac{\triangle t^2}{2} +  u_{ttt}(x_j,t_1^*) \frac{\triangle t^3}{6}.
\end{align*}
Ignoring the third order term, we obtain a second order approximation for the second time step: 
% \begin{align*}
\[
U_{j,1}= U_{j,0} + g(x_j) \triangle t+ s^2 f''(x_j) \frac{\triangle t^2}{2}, \quad 1 \leq j \leq J-1,
\]
or if $f$ is not readily differentiable,
% & U_{j,1}= U_{j,0} + g(x_j) \triangle t+ c^2  \frac{f(x_{j+1}) -2 f(x_{j}) + f(x_{j-1})}{(\triangle x)^2}\frac{\triangle t^2}{2}.
\[
U_{j,1}= U_{j,0} + g(x_j) \triangle t+ \frac{\lambda^2}{2} (f(x_{j+1}) -2 f(x_{j}) + f(x_{j-1})).
% \end{align*}
\]
For this method to be stable, we must have $\lambda \leq 1$.



\begin{problem}
	Consider the initial boundary value problem 
	\begin{align*}
		u_{tt} &= u_{xx}, \\
		u(0,t) &= u(1,t) = 0, \\
		u(x,0) &= \sin(2 \pi x),\\ 
		u_t(x,0) &= 0.
	\end{align*}
	Numerically approximate the solution $u(x,t)$ at $t = .5$. Use $J=5$ subintervals in the $x$ dimension and $M=5$ subintervals in the $t$ dimension. %using $h = .2$, $k = .1$.
	 Compare your results with the analytic solution $u(x,t) = \sin{(2 \pi x)} \cos{(2 \pi t)}.$
\end{problem}



\begin{problem}
	Consider the initial boundary value problem 
	\begin{align*}
		u_{tt} &= u_{xx}, \\
		u(0,t) &= u(1,t) = 0, \\
		u(x,0) &= .2e^{-m^2(x-1/2)^2}\\ 
		u_t(x,0) &= -.4m^2(x-1/2)e^{-m^2(x-1/2)^2}.
	\end{align*}
	The solution of this problem is a Gaussian pulse, and travels to the right at a constant speed. This solution models, for example, a wave pulse in a stretched string. 
	
	Numerically approximate the solution $u(x,t)$ at $t = 1$ (set $m=20$). Use 200 subintervals in space and 220 in time. Then use 200 subintervals in space and 180 in time. Note that the stability condition is not satisfied for the second mesh. 
	% Compare your results with the analytic solution $u(x,t) = \sin{(2 \pi x)} \cos{(2 \pi t)}.$
\end{problem}


% m, u_0 = 20, .2
% def f(x):
% 	return u_0*np.exp(-m**2.*(x-1./2.)**2.)




