\section{Discrete Choice (Threshold) Problems}\label{SecDiscrChoice}

One powerful application of dynamic programming that illustrates its versatility as a dynamic solution method is to models that have both continuous and discrete state variables. These models are sometimes referred to as discrete choice problems or optimal stopping problems. They are also sometimes called threshold problems because the discrete choice policy function is determined by the state variable being above or below certain threshold values. Examples include models of employment that involve both the choice of whether to work and how much to work, models of firm entry and exit that involve the choice of both whether to produce and how much to produce, and models of marriage that involve the choice of whether to date (get married or keep dating) and how much to date.

In this problem set, we follow a simple version of a standard job search model.  Assume that workers are infinitely lived. Let the value of entering a period with most recent wage $w$, current job offer wage $w'$, and employment status $s$ be given by the following value function,
\begin{equation}\label{EqV}
   V(w,w',s) = \begin{cases}
                  V^E(w)    \quad&\text{if}\quad s = E \\
                  V^U(w,w') \quad&\text{if}\quad s = U \\
               \end{cases}
\end{equation}
where employment status is the binary variable $s=\{E,U\}$; a person can be either employed or unemployed.

If an individual's job status is employed ($s = E$) in a given period, then expected utility is the utility of from consumption in the current period plus the discounted expected value of the entering the next period with wage $w$, job offer wage $w''$, and employment status $s'$.
\begin{equation}\label{EqVe1}
   V^E(w) = u(w) + \beta E_{w'',s'}V(w,w'',s')
\end{equation}
Here we assume that individuals spend all of their earnings each period so that utility from the consumption in the current period is $u(w)$.  The discount factor is $\beta$.  The next periods wage, $w''$, and job offer, $s'$, are unknown and we consider them as random variables.  Consequently, the expectations operator $E_{w'',s'}$ is over the job offer wage, and employment status in the next period, and next period's value function is simply \eqref{EqV} with the future value of employment status $s'$.

The joint probability distribution over $w''$ and $s'$ is characterized in the following simple way. If the worker stays employed in the next period $s' = E$, then next period's wage equals the current period's wage. If the worker becomes unemployed in the next period $s' = U$, then the worker's unemployment benefits will be a percentage of his current wage $\alpha w$. Any worker who is unemployed will receive one wage offer per period $w'$, which that worker will receive in the following period, drawn from the cumulative density function $F(w')$ or probability density function $f(w')$, which is independent of the worker's previous wage (for simplicity). Lastly, let $\gamma$ represent the probability that an employed worker becomes unemployed in the next period. So \eqref{EqVe1} can be rewritten in the following way.
\begin{equation}\label{EqVe2}
   V^E(w) = u(w) + \beta \Bigl[(1-\gamma)V^E(w) + \gamma E_{w''}V^U(w,w'')\Bigr]
\end{equation}

The value of being unemployed in a given period is a function of both the wage at the most recent job $w$ as well as the wage of the current job offer $w'$,
\begin{equation}\label{EqVu}
   V^U(w,w') = u(\alpha w) + \beta\max_{s'\in\{E,U\}}\Bigl\{V^E(w'),E_{w''}\left[V^U(w,w'')\right]\Bigr\}
\end{equation}
where $\alpha\in(0,1)$ is the fraction of the worker's previous wage paid in unemployment insurance benefits. It is only in the unemployed state $s=U$ in which the worker makes a decision. Once the job offer is received $w'$ which is drawn from the independent cumulative probability distribution $F(w')$ or the probability density function $f(w')$, the worker can choose whether to accept or reject the offer. The expectation in \eqref{EqVu} is, therefore, not over $w'$ but over the possible job offers in the following period $w''$ if the worker chooses to reject the current job offer $s' = U$.

The policy function for the decision of the unemployed worker whether to accept a job $s'=E$ or whether to reject a job $s'=U$ will be a function of both the amount of the most recent wage $w$ and the amount the the current job offer: $s' = \psi(w,w')$. These discrete choice problems are often called threshold problems because the policy choice depends on whether the state variable is greater than or less than some threshold level. In the labor search model, the threshold level is called the ``reservation wage'' $w_R'$. The reservation wage $w_R'$ is defined as the wage offer such that the worker is indifferent between accepting the job $s' = E$ and staying unemployed $s' = U$.
\begin{equation}\label{EqWR}
   w_R' \equiv w': V^E(w') = E_{w''}\left[V^U(w,w'')\right]
\end{equation}
Note that the reservation wage $w_R'$ is a function of the wage at the most recent job $w$. The policy function will then take the form of accepting the job if $w' \geq w_R'$ or rejecting the job offer and stay unemployed if $w' < w_R'$.
\begin{equation}\label{EqSprime}
   s' = \psi(w,w') = \begin{cases}
                      E \quad\text{if}\quad w' \geq w_R' \\
                      U \quad\text{if}\quad w' < w_R'
                   \end{cases}
\end{equation}

In summary, the labor search discrete choice problem is characterized by the value functions \eqref{EqV}, \eqref{EqVe2}, and \eqref{EqVu}, the reservation wage \eqref{EqWR}, and the policy function \eqref{EqSprime}. Because wage offers are distributed according to the cdf $F(w')$ and because the policy function takes the form of \eqref{EqSprime}, the probability that the unemployed worker receives a wage offer that he will reject is $F(w_R')$ and the probability that he receives a wage offer that he will accept is $1 - F(w_R')$. Just like the continuous choice cake eating problems in problem sets 1 through 5, this problem can be solved by value function iteration, which is similar to starting at the ``final'' period of an individual's life and solving for the an infinite series of solutions by backward induction.

\begin{enumerate}
   
   \item Assume that workers only live a finite number of periods $T$ and assume that the utility of consumption is log utility $u(c) = \log(c)$. The value of entering the last period of life with most recent wage $w$ and employment status $s$ is the following.
   \begin{equation*}
      V_T(w,w',s) = \begin{cases}
                    V^E_T(w) = \log(w) \quad&\text{if}\quad s = E \\
                    V^U_T(w,w') = \log(\alpha w) \quad&\text{if}\quad s = U
                 \end{cases}
   \end{equation*}
   Solve analytically for the value of entering the second-to-last period of life with most recent wage, current job offer, and employment status $V_{T-1}(w,w',s)$ (which includes $V^E_{T-1}(w)$ and $V^U_{T-1}(w,w')$), the reservation wage $w_{R,T-1}'$, and the policy function $s' = \psi_{T-1}(w,w')$.
   
   \item Given the solutions for the $V_{T-1}$, $w_{R,T-1}'$, and $s'=\psi_{T-1}(w')$ from the previous exercise, solve analytically for the value of entering the third-to-last period of life with most recent wage, current job offer, and employment status $V_{T-2}(w,w',s)$ (which includes $V^E_{T-2}(w)$ and $V^U_{T-2}(w,w')$), the reservation wage $w_{R,T-2}'$, and the policy function $s' = \psi_{T-2}(w,w')$. [NOTE: This operation of solving for the new value function $V_t(w,s)$ is a contraction.]
   
\end{enumerate}

The value function iteration solution method for the equilibrium in the labor search problem is analogous to the value function iteration we did in problem sets 3, 4, and 5. The only difference is that two value functions must converge to a fixed point in this problem instead of just one value function converging in the previous problems.

For the following exercises, you will use Python. Assume that the probability of becoming unemployed in a given period is $\gamma = 0.10$, the fraction of wages paid in unemployment benefits is $\alpha = 0.5$, and the discount factor is $\beta = 0.9$. Assume that wage offers to unemployed workers are distributed lognormally $w'\sim \text{LogN}(\mu,\sigma)$ where $m=20$ is the mean wage, $v=400$ is the variance of the wage, $\mu$ is the mean of $\log(w')$ and $\sigma$ is the standard deviation of $\log(w')$. Denote the cdf of the lognormal distribution as $F(w')$ and the pdf of the distribution as $f(w')$.

\vspace{5mm}
\noindent[The following exercises require Python.]
   
\begin{enumerate}
   \setcounter{enumi}{2}
   
   \item Approximate the support of $w\in(0,\infty)$ by generating a column vector of possible values for $w$. Let the maximum value be $w_{max} = 100$, let the minimum value be $w_{min} = 0.2$, and let the number of equally spaced points in the vector be $N = 500$ (an increment value of 0.2). Let the wage of a job offer in any period be lognormally distributed $w'\sim \text{LogN}(\mu,\sigma)$, where $\mu = E\left[\log(w')\right]$ and $\sigma = \sqrt{\text{var}[\log(w')]}$. So if the mean job offer wage $w'$ is $m=20$ and the variance of job offer wages $w'$ is $v=200$, the the corresponding mean $\mu$ and standard deviation $\sigma$ for the lognormal distribution are $\mu=\log\left(m^2/\sqrt{v+m^2}\right)$ and $\sigma=\sqrt{\log\left((v/m^2)+1\right)}$. Generate the discrete approximation of the lognormal probability density function $f(w')$ such that $w'\sim \text{LogN}(\mu,\sigma)$. Thus, $f(w')$ represents the probability of a particular realization $\text{Pr}\left(w'=w_n\right)$. (Hint: This problem is very easy if you use the MatLab function \textit{discretelognorm} in the function file discretelognorm.py available upon request. You're welcome.)
   
   \item Write Python code that solves for the equilibrium optimal policy function $s' = \psi(w,w')$, the reservation wage $w_R'$ as a function of the current wage $w$, and the value functions $V^E(w)$ and $V^U(w,w')$ using value function iteration.
   
   \item Plot the equilibrium reservation wage $w_R'$ of the converged problem as a function of the current wage $w$ with the current wage on the $x$-axis and the reservation wage $w_R'$ on the $y$-axis. This is the most common way to plot discrete choice policy functions. The reservation wage represents the wage that makes the unemployed worker indifferent between taking a job offer and rejecting it. So any wage above the reservation wage line represents $s' = E$ and any wage below the reservation wage line represents $s' = U$.
   
\end{enumerate}
