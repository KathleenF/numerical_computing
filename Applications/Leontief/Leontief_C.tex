\lab{Applications}{Leontief Input-Output Models}{Leontieff Input-Output Models}
\label{Leontief}

\objective{Explain the basics of input-output models, and apply them to analyze a state economy.}

One of the primary applications of linear algebra is modeling interactions between a large number of variables in a concise way.
We can use it to simplify the analysis of very complicated, interactions in a variety of settings.

For example, in economics, one might be interested in how one the inputs of one sector of an economy are reliant upon the outputs of another economy.
To build a car, you need refined steel, plastic, and glass.
These are all outputs of other sectors of the economy.
If there is an increase in car production there will necessarily be increases in production in other sectors of the economy like steel, plastic, and glass.
This type of interaction can be described using what is called a \emph{Leontief Input-Output model}, or just an Input-Output model.

Consider a simple ecomomy divided up into sectors representing agriculture, textiles, and construction.
Given sufficient data, we can estimate how much of the output from one economy is used as an input to another.
An array representing the required inputs in this simple economy is shown in \ref{IOCoefTable}.
This is called a consumption matrix.
It can be especially useful when determining the short term impacts of a given change in the economy.
Directly, its columns represent the inputs required to produce one additional unit of output.
For example, to produce one extra unit of agriculture output we need to input .4 units of agriculture, .04 units of textiles and .08 units of construction.

\begin{table}
\begin{center}
\begin{tabular}{|c|c|c|c|}
\hline
& Agriculture & Textiles & Construction \\ \hline
Agriculture & .4167 & .5357 & .25 \\ \hline
Textiles & .0417 & .0893 & .1667 \\ \hline
Construction & .0833 & .0714 & .0417 \\ \hline
\end{tabular}
\caption{Input Output Coefficients.
The columns represent the amount of each product needed to produce one additional unit of output.}
\label{IOCoefTable}
\end{center}
\end{table}

This sort of matrix can easily be computed from a table of inputs and outputs like the one shown in Table \ref{IORawTable}.
A matrix of input and output values like the one shown in Table \ref{IORawTable} is called an exchange matrix.
If we divide each column by the output of the corresponding industry we can obtain the coefficient matrix (consumption matrix).
We can do this in Python as follows, and we obtain the coefficients in Table \ref{IOCoefTable}.

\begin{lstlisting}
>>> import numpy as np
>>> IO = np.array([[250., 150., 30., 600.], [25., 25., 20., 280.], [50., 20., 5., 120.]])
>>> IOCoeff = IO[:,:3] / IO[:,3]
\end{lstlisting}

\begin{table}
\begin{center}
\begin{tabular}{|c|c|c|c|c|}
\hline
& Agriculture & Textiles & Construction & Total Output \\ \hline
Agriculture & 250 & 150 & 30 & 600 \\ \hline
Textiles & 25 & 25 & 20 & 280 \\ \hline
Construction & 50 & 20 & 5 & 120 \\ \hline
Total Input & 325 & 195 & 55 & \\ \hline
\end{tabular}
\caption{Raw Input Output Data.
Columns represent inputs to an industry, and rows represent outputs.
Notice that not all the total output is used as input for the economy itself.}
\label{IORawTable}
\end{center}
\end{table}

We will now consider the sorts of things these matrices tell us.
By way of example, we will consider the total output needed to produce one extra unit of agriculture.
This certainly costs one unit of agriculture without accounting for the inputs required, but in order to produce one unit of agriculture, we also need inputs of .4167 units of agriculture, .0417 units of agriculture, .0417 units of textiles, and .0833 units of construction.
We also need the inputs necessary to produce those inputs, and the inputs necessary to produce the inputs to the inputs for the inputs, etc.
Letting $d$ be the $3\times 1$ array of needed goods and $C$ be the coefficient matrix, the total cost of producing the outputs in $d$ will be

\[ d + C d + C^2 d + C^3 d + \dots = \left( \sum_{n=1}^{\infty} C^n \right) d = \left( I - C \right)^{-1} d\]

This is just a geometric series in $C$.
The last formula for evaluating the geometric series can be derived in the same way as the formula for a geometric series of complex numbers.
It can be shown that a matrix has an infinite series if and only if the absolute value of each of its eigenvalues is strictly less than one.
This is a constraint that will be built into any real system, and the non-existence of such a series would indicate a problem with the data.

Since the overall effect of the increase in demand from inputs to inputs to inputs, etc. may not happen immediately, it may often be helpful to just evaluate the first two terms in this summation.

\begin{problem}
What would the cost of producing an additional 50\% more units of construction.
Use the formula for the geometric series to find this value.
Notice how a distortion in the market for construction can drastically effect the need for other goods as well.
\end{problem}

We can represent the ``net'' output to consumers $D$ (demand) by the following equation:
\[ D = X - C X \]
In other words, the amout of output used for consumption is the total output minus the portion that is reused as input for the economy itself.

This agrees with our geometric series since the total output of the economy must be the total output required to produce everything that isn't reused, i.e.
\[ X = \left( \sum_{n=1}^{\infty} C^n \right) D = \left( I - C \right)^{-1} D \]
Which is equivalent to $ D = X - C X $.

\begin{problem}
What are the demand levels for the three product economy?
\end{problem}

\begin{problem}
A city is trying to determine how to allocate funds for the celebration of the centennial anniversary of its founding.
It is expecting to take in \$100000 in space (lodging, etc.) \$100000 in consumable goods (food, and other retail items), and \$40000 in services.
Given that the coefficient matrix for space, consumables, and services is
\[ \begin{bmatrix}
.2 & .3 & .3 \\
.1 & .2 & .3 \\
.2 & .2 & .2 \\
\end{bmatrix} \]
Estimate the amount the city should be prepared to produce of each good.
\end{problem}

\begin{problem}
The file \li{WashingtonIOData.npy} contains the raw input output data for the state of Washington in the year 2002.
The last column is the output vector.
The source of this file is the file \li{io2002table.xls}, and you can find the industry corresponding to each column using that file.

Find the output vector if the demand for construction increases by ten percent.
Column 8 corresponds to construction.
This will require first finding the initial demand vector, and then increasing the correct entry by ten percent and using \li{linalg.solve()} to find the new output vector.
\end{problem}

This method can be used to model economies of almost any scale.
One of the primary limitations is that this model assumes that production varies linearly in its inputs, which may not be realistic.
