\lab{Algorithms}{Conservation laws and heat flow}{Conservation laws and heat flow}
\label{lab:finitedifference1}
A conservation law is a balance law, and corresponds to an equation that describes how a quantity is balanced in some system throughout a given process. (Consider how this is related to conservation laws in physics.) For example, suppose we are keeping track of some measurable quantity in a physical system. The fundamental conservation law can be expressed in the following way: the rate of change of the total amount of the quantity in the system equals the rate of the quantity flowing into the system plus the rate that the quantity is produced by sources inside the system. 


\section{Derivation of the Conservation equation in multiple dimensions}
Suppose $\Omega$ is a region in $\mathbb{R}^n$, and $V \subset \Omega$ is bounded with a nice, 
well-behaved boundary $\partial V$. Let $u(\vec{x},t)$ represent the density (concentration) of some quantity throughout $\Omega$. Let $\vec{n}(x)$ represent the normal direction to $V$ at $x \in \partial V$, and let $\vec{J}(\vec{x},t)$ be the flux vector for the quantity, so that $\vec{J}(\vec{x},t) \cdot \vec{n}(x) \, dA$ represents the rate at which the quantity leaves $V$ by crossing a boundary element with area $dA$. Note that the total amount of the quantity in $V$ is 
\[ \int_V u(\vec{x},t)\, dt,\]
and the rate at which the quantity enters $V$ is 
\[-\int_{\partial V} \vec{J}(\vec{x},t) \cdot \vec{n}(x) \, dA.\]

We let the source term be given by $f(\vec{x},t,u)$; we may interpret this to mean that the 
rate at which the quantity is produced in $V$ is 
\[ \int_V f(\vec{x},t,u)\, dt.\]
Then the integral form of the conservation law for $u$ is expressed as 
\[\frac{d}{dt} \int_V u(\vec{x},t) \, d\vec{x} = -\int_V \vec{J}\cdot \vec{n}\, dA + \int_V f(\vec{x},t,u)\, d\vec{x}  .\]

If $u$ and $J$ are sufficiently smooth functions, then we have 
\[ \frac{d}{dt} \int_V u\, d\vec{x} = \int_V u_t \, d\vec{x} ,\]
and 
\[ \int_V \vec{J}\cdot \vec{n}\, dA = \int_V \nabla \cdot \vec{J}\, d\vec{x} .\]
Since this holds for all nice subsets $V \subset \Omega$ with $V$ arbitrarily small, we obtain 
the differential form of the conservation law for $u$: 
\[ u_t + \nabla \cdot \vec{J} = f(\vec{x},t,u) .\]


\section{Constitutive Relations}
Currently our conservation law appears in the form 
\[u_t + \nabla \vec{J} = f(\vec{x},t,u).\]
Thus the conservation law consists of one equation and 2 unknowns ($u$ and $J$).
To this equation we add other equations, called constituve relations, which are used to fully determine the system. 

For example, suppose we wish to describe the flow of heat. Since heat flows from warmer regions to colder regions, and the rate of heat flow depends on the difference in temperature between regions, we usually assume that the flux vector $\vec{J}$ is given by 
\[
\vec{J}(x,t) = -D \nabla u(x,t).
\]
This constitutive relation is called Fick's law, and is the basic model for any diffusive process. Substituting into the conservation law we obtain 
\[u_t -D \triangle u(x,t) = f(\vec{x},t,u).\]  
The function $f$ represents heat sources/sinks within the region. 

\section{Numerically modelling heat flow}
Consider the heat flow equation in one dimension on the interval $[0,1]$ together with appropriate initial conditions and homogeneous Dirichlet boundary conditions: 
\begin{align*}
	u_t &= u_{xx},\\
	u(0,t) &= 0,\\
	u(1,t) &= 0,\\
	u(x,0) &= f(x).
\end{align*}
A simple explicit method for numerically approximating the solution of this problem is to use a forward difference approximation in time and a centered difference approximation in space. Recall that
\[u_t(x_i,t_j) = \frac{u(x_i,t_j + k) - u(x_i,t_j)}{k} + \mathcal{O}(k)
\]
 and 
\[u_{xx}(x_i,t_j) = \frac{u(x_i + h,t_j )-2 u(x_i,t_j)- u(x_i - h,t_j)}{h^2} + \mathcal{O}(h^2).
\]

Let $U_{i,j}$ represent our approximation to $u(x_i,t_j)$. The difference approximations above give us the method 
\begin{align*}
	\frac{U_{i,j+1} - U_{i,j}}{k} &= \frac{U_{i+1,j}- 2U_{i,j} + U_{i-1,j} }{h^2} ,\\
	U_{i,j+1} &= U_{i,j} + \frac{k}{h^2} (U_{i+1,j}- 2U_{i,j} + U_{i-1,j} ).
\end{align*}
This method requires an approximation to the solution at the current step to approximate the solution at the following step.
We can get this method started by using the initial condition given in our problem, so that $U_{i,0} = f(x_i)$. 

Let $f(x) = \sin(2\pi x)$, $h = .2$, and $k=.2$. We will approximate the solution $u(x,t)$ at time $t = 2.$ We need to solve for $U_{i,j}$ on a grid $\{x_i\}_{i=0}^5 \times \{t_j\}_{j=0}^{10}$. Let $U^j = [U_{0,j}, U_{1,j}, U_{2,j}, U_{3,j}, U_{4,j},  U_{5,j}]^T$, so that $U^j$ represents the approximation at time step $j$.
Then 
\[
U^0 = 
\left[\begin{array}{c} 0 \\\sin(.4 \pi) \\\sin(.8 \pi) \\\sin(1.2 \pi) \\\sin(1.6 \pi) \\ 0 \end{array}\right]
\]
and 
\[
U^{j+1} = A U^j + U^j,
% \left[\begin{array}{c}U_{0,j} \\U_{1,j} \\U_{2,j} \\U_{3,j} \\U_{4,j} \\U_{5,j}\end{array}\right]
\]
where $A$ is the tridiagonal matrix given by 
\[
A = \frac{k}{h^2} \left[\begin{array}{cccccc}-2 & 1 &  &  &  &  \\1 & -2 & 1 &  &  &  \\ & 1 & -2 & 1 &  &  \\ &  & 1 & -2 & 1 &  \\ &  &  & 1 & -2 & 1 \\ &  &  &  & 1 & -2\end{array}\right]
\]
Iterating forward in time, we find that 
\[
U^{10} = .
\]

\begin{problem}
	Using the explicit method, numerically approximate the solution $u(x,t)$ of the problem 
	\begin{align*}
		u_t &= u_{xx},\\
		u(0,t) &= 0,\\
		u(1,t) &= 0,\\
		u(x,0) &= \sin(2 \pi x).
	\end{align*}
	Let $h = .05$ and $k= .05$. Then do it for ...(break the CFL condition).
	Animate your results. 
\end{problem}
Mention the CFL condition, and the possibility of a fully implicit method. 
Show the Crank Nicolson method. 


\begin{problem}
	Using the Crank Nicolson method, numerically approximate the solution $u(x,t)$ of the problem 
	\begin{align*}
		u_t &= u_{xx},\\
		u(0,t) &= 0,\\
		u(1,t) &= 0,\\
		u(x,0) &= TODO.
	\end{align*}
	Let $h = .05$ and $k= .05$. 
	Animate your results. 
\end{problem}
%%%%%%%%%%%%%%%%%%%%%%%%%%%%%%%%%%%%%%%%%%%%%%%%%%%%%%%%%%%%%%%%%%%%%%%%%%%%%

% The matrix $A$ is sparse, and so we can use several functions from the package \texttt{scipy.sparse.linalg}. In particular, we use the functions \texttt{spdiags} and \texttt{spsolve}.
% 
% 
% 
% 
% \begin{verbatim}
% D1,D2,D3 = -4*np.ones((1,m**2)), np.ones((1,m**2)), np.ones((1,m**2)) 
% Dm = np.ones((1,m**2))
% for j in range(0,D2.shape[1]):
%     if (j%m)==m-1:
%         D2[0,j]=0
%         if (j%m)==0:
%             D3[0,j]=a0
% diags = np.array([0,-1,1,-m,m])
% data = np.concatenate((D1,D2,D3,Dm,Dm),axis=0)
% 
% A = 1./h**2.*spdiags(data, diags, m**2,m**2).asformat('csr') 
% \end{verbatim}
% 
% \subsection{2D Heat Equation}
% Recall that the collection of finite difference equations
% \[
% \nabla^2_h U_{ij} = 0, \quad 1 \leq i,j\leq m,
% \]
% can be written in matrix form as
% $$AU + q  = 0$$
% 
% 
% The Crank-Nicolson method for the 2D heat equation is given by 
% $$U_{i,\,j}^{n+1}- U_{i,\,j}^{n} = \frac{\Delta t}{2}(\nabla_h^2 U_{i,\,j}^{n} + \nabla_h^2 U_{i,\,j}^{n+1}) \text{ for each } 1 \leq i,j \leq m, $$
% is a second order accurate in both space and time. Basically we're using a midpoint scheme in time, 
% and a trapezoidal scheme in space. The resulting method is implicit, and can be written in matrix form as 
% \begin{align*}
% 	IU^{n+1} &= IU^n + \frac{\Delta t}{2}(AU^n + q + AU^{n+1} + q),\\
% 	(I - \frac{\Delta t}{2}A)U^{n+1}&= (I + \frac{\Delta t}{2}A)U^n + \Delta t q.	
% \end{align*}
% 
% 
% 
% TODO: What size must the time step be to ensure stability? 
% 
% We will need to take many time steps, where many equations must be solved with the matrix $(I - \frac{\Delta t}{2}A)$. The function \texttt{factorized} from \texttt{scipy.sparse.linalg} computes the LU decomposition of the matrix. This decomposition reduces the time required for solving consecutive time steps. 
% 
