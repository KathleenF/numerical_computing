
\lab{Applications}{Monte-Carlo Integration}{Monte-Carlo Integration}

\objective{This section explains the basics of Monte-Carlo Integration. \emph{Application Lab?}}

{\bf Outline:}

\begin{itemize}
\item Background: High Dimensional Integration is hard. Standard methods can have a hard time focusing on important areas. As we add dimensions we have to sample a lot more points (it adds an order of difficulty)
\item Toy Example: Area of a shape? Throwing Darts...
\item Demonstrate how the error decreases ($1/\sqrt{N}$).
\item Talk about adaptive methods? Give example of how to do this in 1D. This will touch on Approximation Theory. Probably better in the Variance Reduction section.
\end{itemize}

\begin{problem}
Calculate a simple 1D problem (maybe using the code for simple newton-cotes from earlier?). Compare convergence with {\tt quad}. Quantify advantages(keep using old data easily). Plot error.
\end{problem}

\begin{problem}
Use MC integration to calculate a higher dimensional integral. Maybe a random many-D high order polynomial, or a function we know should go to zero (high order odd function?)
\end{problem}

{\bf Other possible options (more material/problems)}

\begin{problem}
Explain Asian Options briefly. Show how to calculate their value using MC techniques.
\end{problem}

\begin{problem}
Buffoon's Needle (A simple empirical way to calculate $\pi$) can be formulated as a Monte-Carlo Integration problem. This could be a simulated/experiment lab.
\end{problem}

\begin{problem}
Demonstrate need for good ``random'' variables for MC integration to work (maybe use a delibarately flawed prng on the high-dimensional problem above). One more recent test developed for random variables is based on picking subspaces and testing randomness (I think this is in Knuth), we can talk about how this type of flaw applies to MC integration.
\end{problem}
