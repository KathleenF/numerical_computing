
\lab{Algorithms}{Gaussian Quadrature}{Gaussian Quadrature}
\label{Lab:GaussQuad}
\begin{itemize}
\item In the section on Newton Cotes we saw the difficulty that arises when we use uniformly spaced points (Runge's phenomenon). Can we do better with unevenly spaced points?
\item Explain the gaussian quadrature rule. Explain that it is only appropriate for functions that can be well-approximated by polynomials, and that it is exact for polynomials of sufficiently small degree (not exact in floating point though...)
\item Discuss how we calculate this rule (There's a matrix that we generate, tri-diagonal, of which we need to find the eigenvalues). This is a $O(n^2)$ operation.
\item Explain that the rule as it stands does not allow nesting(and thus adaptive methods), which is crucial. Explain how the Gauss-Konrod rules fix this problem.
\end{itemize}

\begin{problem}
Code up how we find the integration points and the weights.
\end{problem}

\begin{problem}
Code up the actual Gaussian Quadrature method.
\end{problem}

\begin{problem}
This technique can be extended to integrals with other types of weighting functions. Gauss Hermite Quadrature could be interesting to explain and code up, since it involves an infinite domain.
\end{problem}

\begin{problem}
Clenshaw-Curtis quadrature follows a similar approach. Explain that it is based upon finding the roots of the chebyshev polynomial, and deriving the correct weights (which can be done faster than the gaussian quadrature rule). Code and compare performance.
\end{problem}



 
