\lab{Algorithms}{Canonical Transformations}{Canonical Transformations}

\objective{Understand Householder Transformations and Givens Rotations, and explore the reason that they are used in numerical computing.}

{\bf Outline}
\begin{itemize}
\item Explain the reason we use unitary transformations: Very numerically stable. (condition number of one)
\item Explain the mechanics of Householder Transformations: reflecting about a plane. Exercise 3.9 in book.
\item Explain the mechanics of Givens Rotations: rotating on a plane.
\item Use Householder Transformations to calculate QR decomposition. Compare stability.
\item Explain what the Upper Hessenberg Form is
\end{itemize}

\begin{problem}
Write a script that transfers an input matrix to upper hessenberg form. We will use this technique in the eigenvalue lab later.
\end{problem}

\begin{problem}
Use givens rotations to do the QR decomposition. Which form (mgs, Householder or Givens) is fastest? Which is most stable? Explain that the Givens rotations are not as fast as Householder in the general case, but in the sparse case they are faster, and are very parallelizable. Explain also that MGS is important for some types of iterative methods, because it finds the orthogonal basis one vector at a time instead of all at once.
\end{problem}

%Sources: http://www.cs.unc.edu/~krishnas/eigen/node5.html
% http://en.wikipedia.org/wiki/Givens_rotation
%http://en.wikipedia.org/wiki/QR_decomposition
%	Note the Operation count: Householder is 2/3 n^3, MGS is 2 n^3
%http://en.wikipedia.org/wiki/QR_algorithm
%Applied Numerical methods using MATLAB by Yang has some code written for this
%http://www.math.kent.edu/~reichel/courses/intr.num.comp.2/lecture21/evmeth.pdf
%	These are eigenvalue algorithms explained carefully
%http://en.wikipedia.org/wiki/Householder_transformation

