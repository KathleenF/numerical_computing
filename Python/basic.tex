\chapter{The Python Language}{The Python Language}
\label{Lab:PythonBasic}

In this lab we will focus on the bas ic constructs of the Python programming language.  This will serve as a basic introduction in using the core language.  Understanding this lab will be essential to your success in the following labs.

\section*{What is Python}
Python is a very powerful interpreted language.  

Python is an interpreted language.  The advantage of this is quick development time and fast debugging.  However, speed is traded for these benefits.  When Python code is executed, it is translated into bytecode and run on a virtual machine.

Let us look a small example of Newton's Method.  We wrote a very na\"ive implementation of Newton's Method.
\begin{lstlisting}[style=python]
def NewtonsMethod(f, df, x0, tol=1e-7):
    while abs(f(x0)) >= tol:
        x0 -= float(f(x0))/df(x0)
    return x0
\end{lstlisting}
When executing this, the python interpreter first translates it into Python bytecode which is then executed by a virtual machine.  Often, this bytecode is stored as a \li{.pyc} file in the same directory.  Python understands our Newton's Method function as (the numbers on the far left correspond to line numbers):
\begin{verbatim}
  2           0 SETUP_LOOP              60 (to 63)
        >>    3 LOAD_GLOBAL              0 (abs)
              6 LOAD_FAST                0 (f)
              9 LOAD_FAST                2 (x0)
             12 CALL_FUNCTION            1
             15 CALL_FUNCTION            1
             18 LOAD_FAST                3 (tol)
             21 COMPARE_OP               5 (>=)
             24 POP_JUMP_IF_FALSE       62

  3          27 LOAD_FAST                2 (x0)
             30 LOAD_GLOBAL              1 (float)
             33 LOAD_FAST                0 (f)
             36 LOAD_FAST                2 (x0)
             39 CALL_FUNCTION            1
             42 CALL_FUNCTION            1
             45 LOAD_FAST                1 (df)
             48 LOAD_FAST                2 (x0)
             51 CALL_FUNCTION            1
             54 BINARY_DIVIDE       
             55 INPLACE_SUBTRACT    
             56 STORE_FAST               2 (x0)
             59 JUMP_ABSOLUTE            3
        >>   62 POP_BLOCK           

  4     >>   63 LOAD_FAST                2 (x0)
             66 RETURN_VALUE
\end{verbatim}

\section*{Why Python?}
\begin{itemize}
\item Python is a robust, versitile object-oriented language.  It aims toward code that is easily written and more easily read and maintained.  The language is manages its own memory and is scalable and portable.  Python programs can run on any platform on which a Python interpreter can run.  Programs written on Windows, will run on Linux and Mac OSX, with little or no changes.
\item Interactive development model that allow fast prototyping and development of algorithms.
\item Many mature and very efficient scientific computing libraries exist for Python.  Numpy and Scipy provide many of the fast algorithms available in MATLAB.  Matplotlib is a mature plotting library that is capable of producing a wide range of plots, graphs, and other visualizations.
\item Python has a large, active community.  It is used in many 
\item Free
\end{itemize}


\section*{Resources for Learning Python}
There already exist many well written introductions to the Python language.  We recommend you review the following sources if you are learning Python for the first time.

%List of Python Resources
Scipy Lecture Notes
http://scipy-lectures.github.com/

Dive into Python
Version 2: http://www.diveintopython.net/toc/index.html
Version 3: http://getpython3.com/diveintopython3/



\section*{Effiencent Workflows}
An efficient workflow is absolutely essential in scientific computing.  Few people wish to hindered by the tools they use.  In this section, we will propose a couple of workflows that have proven to work well.

\subsection*{Text editor + Python Interpreter/IPython}
This is the most basic workflow to use.  It involves using your favorite text editor to edit source files and then executing those scripts in your command terminal.  Some popular editors for python source code include
\begin{itemize}
\item Emacs
\item Vim
\item Notepad++ (Windows)
\end{itemize}

Typically, a working directory is created where scripts can be saved.

To run a script in the Python Interpeter, execute the following (for IPython, replace \li{python} with \li{ipython})
\begin{verbatim}
$ python myscript.py
\end{verbatim}

The Python interpreter can be used interactively.  In the interactive mode, statements will be executed one at a time as you enter them.  It is extremely useful and allows for very rapid programming.

\subsection*{Integrated Development Environments}
Integrated Development Environments provide a comprehensive environment for application development.  Most IDEs have many tightly integrated tools that are easily accessible.  Some popular IDEs for developing in Python are discussed in this section.

\subsubsection*{Eclipse + PyDev}

Eclipse: \url{http://www.eclipse.org/} \\
PyDev: \url{http://pydev.org/}

Eclipse is a general purpose IDE that supports many languages.  The PyDev plugin for Eclipse contains all the tools needed to start working in Python.  It includes a built-in debugger, and has a very nice code editor.  Eclipse + PyDev is available for Windows, Linux, and Mac OSX.

\subsubsection*{Spyder}

Spyder: \url{http://code.google.com/p/spyderlib/}

Spyder is a Python IDE that is designed specifically for scientific computing.  Its interface is reminescent of MATLAB.  Requires PyQT4 to be installed.  Spyder is available for Windows and Linux.  There is a version available for Mac OSX through MacPorts.

\subsubsection*{IPython Notebook}
The IPython Notebook is a web-based interface to Python.


% Try focusing on the underpinnings of Python:
% Interpreter/Instructions
% Execution Flow (Script, interpreter, bytecode, execution)
% Explain packages and libraries
