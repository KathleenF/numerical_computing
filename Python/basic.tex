\chapter{The Python Language}{The Python Language}
\label{Lab:Python}

In this lab we will focus on the bas ic constructs of the Python programming language.  This will serve as a basic introduction in using the core language.  Understanding this lab will be essential to your success in the following labs.

\section*{Resources for Learning Python}
There already exist many well written introductions to the Python language.  We recommend you review the following sources if you are learning Python for the first time.

%List of Python Resources

\section*{Effiencent Workflows}
An efficient workflow is absolutely essential in scientific computing.  Few people wish to hindered by the tools they use.  In this section, we will propose a couple of workflows that have proven to work well.

\subsection*{Text editor -> Python Interpreter}
This is the most basic workflow to use.  It involves using your favorite text editor to edit source files and then executing those scripts in your command terminal.

\subsection*{Text editor -> IPython}
\subsection*{Integrated Development Environments}
\subsubsection*{Eclipse}
\subsubsection*{Spyder}

\section*{Functions}
Functions are reusable blocks of code that can perform one or more tasks.  A function can accept zero or more parameters called arguments.  Some parameters can have default values and are referred to as keyword arguments.
Pass by value vs pass by reference.  How python passes references LATER in the advanced tutorial

\section*{Scripts}
The labs that follow will ask you write many scripts.  At the most basic level, a script is simply a file containing a set of functions.

\section*{Libraries}


Try focusing on the underpinnings of Python:
Interpreter/Instructions
Execution Flow (Script, interpreter, bytecode, execution)
Explain packages and libraries
