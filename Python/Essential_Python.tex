\lab{Essentials}{Python}{Python}
\label{lab:Essentials1}

Python is a very powerful general-purpose programming language.  It is quickly gaining momentum as a tool in scientific computing.  Python has several very nice key features
\begin{itemize}
\item Clear, readable syntax
\item Full object orientation
\item Complete memory management (via garbage collection)
\item High level, dynamic datatypes
\item Extensible via C or C++
\item Embeddable in applications
\item Portable across many platforms (Linux, Windows, Mac OSX)
\end{itemize}
In addition to these, Python is freely available and can be freely distributed.

\section*{Installing Python and other libraries}
Let us set up a basic environment.  We will be installing four programs that are required to do the labs and an additional program that will greatly simply your Python experience.
\begin{enumerate}
\item Python 2.7 \url{http://www.python.org/}
\item NumPy \url{http://www.numpy.org/}
\item SciPy \url{http://www.scipy.org/}
\item Matplotlib \url{http://matplotlib.sourceforge.net/}
\item IPython \url{http://ipython.org/} (Optional)
\end{enumerate}
You may install the programs by yourself or if you wish, you may download and install one of the Python distributions below which will include all these libraries.
\begin{itemize}
\item EPD Free by Enthought (\url{http://www.enthought.com/products/epd_free.php})
\item Python(x,y) (\url{http://code.google.com/p/pythonxy/})
\end{itemize}


\subsection*{Python 2.7}
Python can easily be downloaded and installed from \url{http://www.python.org/}.  For the labs in this book, Python 2.7 is required.  Python 3 will not work because matplotlib, the plotting library required in some of the labs is not compatible with Python 3.  Please download the version that is appropriate for your system.  Note that for Mac OSX, you will need to install Python because the version of Python included with OSX a very minimal version of Python.

\subsection*{NumPy}
Download and install the latest NumPy from the website.  Be sure to download the installer that is compatible with Python 2.7.

\subsection*{SciPy}
Download and install the latest SciPy from the website.  Be sure to download the installer that is compatible with Python 2.7.

\subsection*{Matplotlib}
Download and install the latest SciPy from the website.  Be sure to download the installer that is compatible with Python 2.7.

\subparagraph*{IPython}
This package requires another package called \texttt{readline}.  This is already installed on most Linux systems.  For Windows, PyReadline is available at \url{http://pypi.python.org/pypi/pyreadline}.  For Mac OSX, \texttt{readline} is available at \url{http://pypi.python.org/pypi/readline}

IPython can be installed via Python.  Open up a command window and run
\begin{lstlisting}
$ pip install ipython
\end{lstlisting}

This should automatically download and configure IPython for you.

\section*{Workflows}
There are several different ways to write your programs and run them in Python.

\section*{Introducing Python}
There already exist many well written introductions to the Python language.  We recommend you review the following sources if you are learning Python for the first time.

\begin{itemize}
\item Official Python Tutorial: \url{http://docs.python.org/tutorial/index.html}
\item SciPy Lecture Notes: \url{http://scipy-lectures.github.com/}
\item Dive Into Python 2: \url{http://www.diveintopython.net/toc/index.html}
\item Dive Into Python 3: \url{http://getpython3.com/diveintopython3/}
\item Python Style Guide: \url{http://www.python.org/dev/peps/pep-0008/}
\end{itemize}

\section*{Using the Python Interpreter}
Python is an interpreted language.  One advantage of this is that programs written in Python are platform independent.  Python source code is translated to bytecode and then executed by a virtual machine.  A Python script written on Windows will work exactly the same on Linux or Mac OSX.  There is no need to compile a Python script.  It is only necessary to feed it into the Python interpreter.  Let's illustrate with a small example.  We will demonstrate the development of a simple algorithm, the Euclidean Algorithm.  Type the following into your text editor of choice
\lstinputlisting[style=fromfile]{euclidean.py}

\begin{lstlisting}
$ python euclidean.py
\end{lstlisting}

