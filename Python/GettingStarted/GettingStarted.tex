\lab{Python}{Getting Started}{Getting Started}
\label{lab:Essential_Python}

Python is a very powerful general-purpose programming language. 
It is quickly gaining momentum as a tool in scientific computing.  
Python has several very nice key features
\begin{itemize}
\item Clear, readable syntax
\item Full object orientation
\item Complete memory management (via garbage collection)
\item High level, dynamic datatypes
\item Extensible via C or C++
\item Embeddable in applications
\item Portable across many platforms (Linux, Windows, Mac OSX)
\end{itemize}
In addition to these, Python is freely available and can be freely distributed.

\section*{Installing Python and other libraries}
Let us set up a basic environment.  
We will be installing four programs that are required to do the labs and an additional program that will greatly simply your Python experience.
\begin{enumerate}
\item Python 2.7 \url{http://www.python.org/}
\item NumPy \url{http://www.numpy.org/}
\item SciPy \url{http://www.scipy.org/}
\item Matplotlib \url{http://matplotlib.sourceforge.net/}
\item IPython \url{http://ipython.org/} (Optional)
\end{enumerate}
You may install the programs by yourself or if you wish, you may download and install one of the Python distributions below which will include all these libraries.  
For Mac OSX systems, it is recommended to install the Enthought Python Distribution.
\begin{itemize}
\item EPD Free by Enthought (\url{http://www.enthought.com/products/epd_free.php}) \emph{Windows, Linux, Mac OSX}
\item Python(x,y) (\url{http://code.google.com/p/pythonxy/}) \emph{Windows}
\end{itemize}

\subsection*{Python 2.7}
Python can easily be downloaded and installed from \url{http://www.python.org/}.  
For the labs in this book, Python 2.7 is required.
Please download the version that is appropriate for your system.  
Note that for Mac OSX, you will need to install Python because the version of Python included with OSX a very minimal version of Python.

\subsection*{NumPy}
Download and install the latest NumPy from \url{http://www.numpy.org/}.  
Be sure to download the installer that is compatible with Python 2.7.

\subsection*{SciPy}
Download and install the latest SciPy from \url{http://www.scipy.org/}. 
Be sure to download the installer that is compatible with Python 2.7.

\subsection*{Matplotlib}
Download and install the latest SciPy from \url{http://matplotlib.sourceforge.net/}.
Be sure to download the installer that is compatible with Python 2.7.

\subparagraph*{IPython}
This package requires another package called \texttt{readline}.
This is already installed on most Linux systems. 
For Windows, PyReadline is available at \url{http://pypi.python.org/pypi/pyreadline}.
For Mac OSX, \texttt{readline} is available at \url{http://pypi.python.org/pypi/readline}

IPython can be installed via Python.
Open up a command window and run
\begin{lstlisting}
$ pip install ipython
\end{lstlisting}
This should automatically download and configure IPython for you.

\section*{Workflows}
There are several different ways to write your programs and run them in Python.

\subsection*{Text editor + Python Interpreter/IPython}
This is the most basic workflow to use.  
It involves using your favorite text editor to edit source files and then executing those scripts in your command terminal.  
Some popular editors for python source code include
\begin{itemize}
\item Emacs
\item Vim
\item Notepad++ (Windows)
\item TextWrangler (OS X)
\end{itemize}
Typically, a working directory is created where scripts can be saved.

To run a script in the Python Interpeter, execute the following (for IPython, replace \li{python} with \li{ipython})
\begin{verbatim}
$ python myscript.py
\end{verbatim}
The Python interpreter can be used interactively. 
In the interactive mode, statements will be executed one at a time as you enter them. 
It is extremely useful and allows for very rapid programming.

IPython is an enhanced interactive interpreter for Python.
It has many features that cater to productivity.  One very useful feature is 
object introspection.  IPython allows you to examine the properties and methods
of any object in the interpreter.  Another useful feature is tab completion.
The IPython interpreter and a text editor is often a preferred combination.

\subsection*{Integrated Development Environments}
Integrated Development Environments provide a comprehensive environment for application development. 
Most IDEs have many tightly integrated tools that are easily accessible. 
Some popular IDEs for developing in Python are discussed in this section.

\subsubsection*{Eclipse + PyDev}

Eclipse: \url{http://www.eclipse.org/} \\
PyDev: \url{http://pydev.org/}

Eclipse is a general purpose IDE that supports many languages.  
The PyDev plugin for Eclipse contains all the tools needed to start working in Python.
It includes a built-in debugger, and has a very nice code editor. 
Eclipse + PyDev is available for Windows, Linux, and Mac OSX.

\subsubsection*{Spyder}

Spyder: \url{http://code.google.com/p/spyderlib/}

Spyder is a Python IDE that is designed specifically for scientific computing. 
Its interface is reminescent of MATLAB. 
Requires PyQT4 to be installed. 
Spyder is available for Windows and Linux.  
There is a version available for Mac OSX through MacPorts.

\subsubsection*{IPython Notebook}
The IPython Notebook is a web-based interface to Python.

\section*{Introducing Python}
There already exist many well written introductions to the Python language. 
We recommend you review the following sources if you are learning Python for the first time.

\begin{itemize}
\item Official Python Tutorial: \url{http://docs.python.org/tutorial/index.html}
\item SciPy Lecture Notes: \url{http://scipy-lectures.github.com/}
\item Dive Into Python 2: \url{http://www.diveintopython.net/toc/index.html}
\item Dive Into Python 3: \url{http://getpython3.com/diveintopython3/}
\item Python Style Guide: \url{http://www.python.org/dev/peps/pep-0008/}
\end{itemize}
If time is short, some beneficial topics to focus on in your study are
\begin{itemize}
\item Lists, sets, tuples, and dictionaries
\item Loops
\item Functions
\end{itemize}


\section*{Using the Python Interpreter}
Python is an interpreted language.
One advantage of this is that programs written in Python are platform independent. 
Python source code is translated to bytecode and then executed by a virtual machine.
A Python script written on Windows will work exactly the same on Linux or Mac OSX.
There is no need to compile a Python script.
It is only necessary to pass the name of the script to the Python interpreter. 
Let's illustrate with a small example.
We will demonstrate the development of a simple algorithm, the Euclidean Algorithm.
Type the following into your text editor of choice
\lstinputlisting[style=fromfile]{euclidean.py}

\begin{lstlisting}
$ python euclidean.py
\end{lstlisting}
Let's use the interpreter interactively. 
We enter the interactive interpreter by invoking python without any arguments
\begin{lstlisting}
$ python
Python 2.7.3 (default, May 29 2012, 12:55:35) 
[GCC 4.7.0 20120505 (prerelease)] on linux2
Type "help", "copyright", "credits" or "license" for more information.
>>> 
\end{lstlisting}
Or, alternatively, we can use the IPython interpreter.  
For these labs, we recommend the use of IPython instead of the default Python interpreter.
\begin{lstlisting}
$ ipython
Python 2.7.3 (default, May 29 2012, 12:55:35) 
Type "copyright", "credits" or "license" for more information.

IPython 0.13 -- An enhanced Interactive Python.
?         -> Introduction and overview of IPython's features.
%quickref -> Quick reference.
help      -> Python's own help system.
object?   -> Details about 'object', use 'object??' for extra details.

In [1]: 
\end{lstlisting}
When we are in interactive mode, we can input Python commands and expressions directly into the interpreter and they will be immediately executed. 
This is very useful for experimenting with an algorithm or other ideas.

\begin{problem}
Familiarize yourself with the IPython interpreter.
Read the Quick Reference by typing \li{\%quickref} in the IPython interpreter.

IPython will be our preferred Python interpreter.  Take time to configure your own Python
workflow.  Each have their own strengths.  A tool is only useful in the hands of somebody
who knows how to effectively use it.
\end{problem}
