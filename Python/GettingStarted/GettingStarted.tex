\lab{Python}{Getting Started}{Getting Started}
\label{lab:Essential_Python}

Python is a powerful general-purpose programming language. 
It is quickly gaining momentum as a tool in scientific computing and has several very nice key features:
\begin{itemize}
\item Clear, readable syntax
\item Full object orientation
\item Complete memory management (via garbage collection)
\item High level, dynamic datatypes
\item Extensibility via C
\item Interfaces to other languages like R, C, C++, and Fortran
\item Embeddable in applications
\item Portable across many platforms (Linux, Windows, Mac OSX)
\item Open source
\item Free availability
\end{itemize}
In addition to these, Python is freely available and can also be freely distributed. 

\section*{Installing Python and other libraries}
Begin by installing the following programs:
\begin{enumerate}
\item Python 2.7 or above\url{http://www.python.org/}
\item NumPy 1.7.0 or above\url{http://www.numpy.org/}
\item SciPy 0.12.0 or above\url{http://www.scipy.org/}
\item Matplotlib 1.2.1 or above \url{http://matplotlib.sourceforge.net/}
\item IPython 0.13.2 or above\url{http://ipython.org/}
%also consider including
%cython, f2py, and sympy
\end{enumerate}
You may install the programs by yourself or download and install one of the Python distributions below which will include all these libraries.
The first two work on Linux, Macintosh, and Windows.
Python(x,y) is just for windows.
EPD and Anaconda provide free academic licenses which give you access to some highly optimized matrix math libraries.
\begin{itemize}
\item Anaconda Python Distribution (\url{http://www.continuum.io/downloads})
\item EPD Free by Enthought (\url{http://www.enthought.com/products/epd_free.php}) \emph{Windows, Linux, Mac OSX}
\item Python(x,y) (\url{http://code.google.com/p/pythonxy/}) \emph{Windows}
\end{itemize}

For Windows 64 bit, we recommend you follow our install guide included as lab \ref{win64install}.

\subsection*{Python 2.7}
Python can easily be downloaded and installed from \url{http://www.python.org/}.
For Windows we recommend that you use one of the distributions listed above instead of building all the packages from source.
For the labs in this book, Python 2.7 is required.
Please download the version that is appropriate for your system.
Note that for Mac OSX, you will still need to install Python.
The version that is included with OSX is too limited for our purposes here.

\subsection*{NumPy}
Download and install the latest NumPy from \url{http://www.numpy.org/}.  
Be sure to download the installer that is compatible with Python 2.7.

\subsection*{SciPy}
Download and install the latest SciPy from \url{http://www.scipy.org/}. 
Be sure to download the installer that is compatible with Python 2.7.

\subsection*{Matplotlib}
Download and install the latest SciPy from \url{http://matplotlib.sourceforge.net/}.
Be sure to download the installer that is compatible with Python 2.7.

\subsection*{IPython}
This package requires another package called \texttt{readline}.
This is already installed on most Linux systems. 
For Windows, PyReadline is available at \url{http://pypi.python.org/pypi/pyreadline}.
For Mac OSX, \texttt{readline} is available at \url{http://pypi.python.org/pypi/readline}

IPython can be installed via Python.
Open up a command window and run
\begin{lstlisting}
$ pip install ipython
\end{lstlisting}
This should automatically download and configure IPython for you.
IPython also includes a useful notebook interface to Python.
If you would like to use the notebook, you will also need to download and install the packages \li{Tornado} and \li{pyzmq}.
These can be found at \url{https://pypi.python.org/pypi/pyzmq} and \url{https://pypi.python.org/pypi/tornado/3.0.1}.
Both can be installed using \li{pip} in the command line as above.

If you have trouble installing packages on Windows, a good resource is \url{http://www.lfd.uci.edu/~gohlke/pythonlibs/} where you can download pre-compiled windows installers for many different Python packages.

\section*{Workflows}
There are several different ways to write your programs and run them in Python.

\subsection*{Text editor + Python Interpreter/IPython}
This is the most basic workflow to use.  
It involves using your favorite text editor to edit source files and then executing those scripts in your command terminal.  
Some popular editors for python source code include
\begin{itemize}
\item Emacs
\item Vim
\item Notepad++ (Windows)
\item TextWrangler (OS X)
\end{itemize}
Typically, a working directory is created where scripts can be saved.

To run a script in the Python Interpeter, execute the following (for IPython, replace \li{python} with \li{ipython})
\begin{verbatim}
$ python myscript.py
\end{verbatim}
The Python interpreter can be used interactively. 
In the interactive mode, statements will be executed one at a time as you enter them. 
It is extremely useful and allows for very rapid programming.

IPython is an enhanced interactive interpreter for Python.
It has many features that cater to productivity.
One very useful feature is object introspection.
This feature allows you to examine the properties and methods of any object in the interpreter.
Another useful feature is tab completion.
Using the IPython interpreter with a text editor is a good way to work in Python.

\subsection*{Integrated Development Environments}
Integrated Development Environments provide a comprehensive environment for application development. 
Most IDEs have many tightly integrated tools that are easily accessible. 
Some popular IDEs for developing in Python are discussed in this section.

\subsubsection*{Eclipse + PyDev}

Eclipse: \url{http://www.eclipse.org/} \\
PyDev: \url{http://pydev.org/}

Eclipse is a general purpose IDE that supports many languages.  
The PyDev plugin for Eclipse contains all the tools needed to start working in Python.
It includes a built-in debugger, and has a very nice code editor. 
Eclipse + PyDev is available for Windows, Linux, and Mac OSX.

\subsubsection*{Spyder}

Spyder: \url{http://code.google.com/p/spyderlib/}

Spyder is a Python IDE that is designed specifically for scientific computing. 
Its interface is reminescent of MATLAB. 
Requires PyQT4 to be installed. 
Spyder is available for Windows and Linux.  
There is a version available for Mac OSX through MacPorts.

\subsubsection*{IPython Notebook}
The IPython Notebook is a browser-based interface to Python.
It is similar to the notebook interfaces used by Mathematica, Maple, and Sage.
Running a notebook is similar to running a Python interpreter session except that the input is stored in cells and can be modified and re-evaluated as needed.
You can also save notebooks and reload them later.
The IPython notebook is included as part of the Anaconda Python Distribution, the Enthought Python Distribution, and Python(x,y).

\section*{Introducing Python}
There already exist many well written introductions to the Python language. 
We recommend you review the following sources if you are learning Python for the first time.

\begin{itemize}
\item Official Python Tutorial: \url{http://docs.python.org/tutorial/index.html}
\item SciPy Lecture Notes: \url{http://scipy-lectures.github.com/}
\item Dive Into Python 3: \url{http://getpython3.com/diveintopython3/}
\item Python Style Guide: \url{http://www.python.org/dev/peps/pep-0008/}
\end{itemize}
If time is short, some beneficial topics to focus on in your study are
\begin{itemize}
\item Lists, sets, tuples, and dictionaries
\item Loops
\item Functions
\end{itemize}

\begin{problem}
\begin{itemize}
\item Read chapters 3, 4, and 5 of the Official Python Tutorial.
\item Read sections 1.2 and 2.1 of the SciPy Lecture Notes.
\item Read the Python Style Guide (PEP8).
\end{itemize}
\end{problem}

\section*{Using the Python Interpreter}
Python is an interpreted language.
One advantage of this is that programs written in Python are platform independent. 
Python source code is translated into bytecode and then executed by a virtual machine.
A Python script written on Windows will work exactly the same on Linux or Mac OSX.
There is no need to compile a Python script.
It is only necessary to pass the name of the script to the Python interpreter. 
Let's illustrate with a small example.
We will demonstrate the development of a simple algorithm, the Euclidean Algorithm.
Type the following into your text editor of choice and save it in your usual working directory as \li{euclidean.py}
\lstinputlisting[style=fromfile]{euclidean.py}
This program can be run by running the following command in your terminal:
\begin{lstlisting}
$ python euclidean.py
\end{lstlisting}

Let's use the Python interpreter interactively.
This is done by running the command \li{python} in your terminal.
The output should be something like this:
\begin{lstlisting}
$ python
Python 2.7.3 (default, May 29 2012, 12:55:35) 
[GCC 4.7.0 20120505 (prerelease)] on linux2
Type "help", "copyright", "credits" or "license" for more information.
>>> 
\end{lstlisting}
Or, alternatively, we can use the IPython interpreter.  
For these labs, we recommend the use of IPython instead of the default Python interpreter.
The IPython interpreter is started by running the command \li{ipython} in your terminal.
When it starts, the output should look like this:
\begin{lstlisting}
$ ipython
Python 2.7.5 (default, May 12 2013, 12:00:47) 
Type "copyright", "credits" or "license" for more information.

IPython 0.13.2 -- An enhanced Interactive Python.
?         -> Introduction and overview of IPython's features.
%quickref -> Quick reference.
help      -> Python's own help system.
object?   -> Details about 'object', use 'object??' for extra details.

In [1]: 
\end{lstlisting}
When we are running Python in interactive mode, we can input Python commands and expressions directly into the interpreter and they will be immediately executed. 
This is very useful for experimenting with an algorithm or other ideas.

IPython will be our preferred Python interpreter.  Take time to configure your own Python
workflow.  Each have their own advantages.  A tool is only useful in the hands of somebody
who knows how to effectively use it.
\begin{problem}
Familiarize yourself with the IPython interpreter.
Read the Quick Reference by typing \li{\%quickref} in the IPython interpreter.

If you decided to install the IPython Notebook, verify that it works by running small blocks of Python code.
The notebook can be started with \li{ipython notebook} in your terminal.
A cell in a notebook is evaluated by pressing \li{shift+enter} while your cursor is in the cell you want to evaluate.
\end{problem}
