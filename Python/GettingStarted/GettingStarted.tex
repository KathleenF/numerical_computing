\lab{Python}{Getting Started}{Getting Started}
\label{lab:Essential_Python}

Python is a powerful general-purpose programming language. It is an interpreted
language and can even be used interactively. 
It is quickly gaining momentum as a tool in scientific computing and has several very
nice key features:
\begin{itemize}
\item Clear, readable syntax
\item Full object orientation
\item Complete memory management (via garbage collection)
\item High level, dynamic datatypes
\item Extensibility via C
\item Interfaces to other languages like R, C, C++, and Fortran
\item Embeddable in applications
\item Portable across many platforms (Linux, Windows, Mac OSX)
\item Open source

\end{itemize}
In addition to these, Python is freely available and can also be freely distributed.
For more information on installation and other libraries, check out the Appendix.


\section*{Introducing Python}
There are many well written introductions to the Python language.
We recommend you review the following sources if you are learning Python for the
first time.

\begin{itemize}
\item Official Python Tutorial: \url{http://docs.python.org/tutorial/index.html} \\
      Start with Chapter 3: \url{http://docs.python.org/2/tutorial/introduction.html}
\item SciPy Lecture Notes: \url{http://scipy-lectures.github.com/}
\item Dive Into Python 3: \url{http://getpython3.com/diveintopython3/}
\item Python Style Guide: \url{http://www.python.org/dev/peps/pep-0008/}
\end{itemize}

The following examples and problems highlight important functionality
and syntax in Python, but are clearly not exhaustive.
For further information we highly recommend you read the following.
\begin{enumerate}
\item Chapters 3, 4, and 5 of the Official Python Tutorial \\
        (\url{http://docs.python.org/2/tutorial/introduction.html}).
\item Section 1.2 of the SciPy Lecture Notes \\
        (\url{http://scipy-lectures.github.io/}).
\item PEP8 - Python Style Guide \\
        (\url{http://www.python.org/dev/peps/pep-0008/}).
\end{enumerate}

Note that comments in Python begin with the hash character \li{#} 
and extend to the end of the line. Comments that require multiple
lines are multi-line strings that can be enclosed by \li{'''} or \li{"""} at the beginning 
and end of your comment. 


It is suggested that you open your command line or the IPython Notebook 
and work through the following examples to verify. 

\subsection*{Numbers and Strings}

\begin{example}
Python can be used as a calculator. It also 
includes the \li{string} module.

\begin{lstlisting}
In [1]: #this is a comment!

In [2]: 9 * 8
Out[2]: 72

In [3]: 100/3 #integer division returns the floor.
Out[3]: 33

In [4]: 100/3.0 #this converts the integer operand to a floating point operand. When at least one of the values is a float, the result is also a float. 
Out[4]: 33.333333333333336

In [5]: x = 12

In [6]: y = 3 * 4

In [7]: x * y
Out[7]: 144

In [8]: a, b, c = 1, 2, 3 #this is called multiple assignment

In [9]: my_string = "I love the new ACME program!" #this is a string

In [10]: my_string[15:19] #strings can be sliced to access a specific portion of the string.
Out[10]: 'ACME'

In [11]: my_string[:6] #or to get the first 6 characters. This is equivalent to my_string[0:6]
Out[11]: 'I love'

In [12]: my_string[::2] #or every other character! #SYNTAX: my_string[start:stop:step]
Out[12]: 'Ilv h e CEporm'

\end{lstlisting}
\end{example}

\begin{problem}
Answer the following questions with the best answer:
\begin{itemize}
\item Why does \li{7/3} return \li{2} in Python 2? 
\item What are the two ways to create a complex number? 
How do you extract just the real part and just the imaginary part?
\item How would you cast an integer as a float?
\item Is there a way to explicitly express integer division when using floats?
\item What does it mean for \li{string} to be an immutable object? 
\item What happens in \li{string[::2]} and \li{string[27:0:-1]} provided 
that \li{string = "I love the new ACME program\!"}? 
How can you access the entire string in reverse?

\end{itemize}
\end{problem}


\subsection*{Containers}

\begin{example}
Python contains a variety of container datatypes like \li{list}, \li{set}, 
\li{dict}, and \li{tuple}.

\begin{lstlisting}
In [1]: my_list = ["Remi", 21, "02/05", 1993] #lists are written as comma-separated items 

In [2]: my_list[0] #lists are indexed starting at zero.
Out[2]: 'Remi'

In [3]: my_list[2] = "February 05"

In [4]: len(my_list) #returns the number of items in the list.
Out[4]: 4

In [5]: basket = set(["banana", "sandwich", "apple"]) #a set is an unordered collection with no duplicate elements. 

In [6]: "banana" in basket
Out[6]: True

In [7]: # List comprehensions provide a concise way to create lists, which are mutable sequences. 

In [8]: squares = [x**2 for x in range(10)] # this creates a list of squares. SYNTAX: range(start, stop, step) produces [start, start + step, start + 2* step, ...]

In [9]: tel = {"accounting":4234, "admissions": 2507, "financial aid": 4104, "marriott": 4121, "math": 2061, "visual arts" : 7321} #dictionaries are unordered sets of key:value pairs. Note that keys are unique.

In [10]: tel["math"]
Out[10]: 2061

In [11]: my_tuple = 123, 456, 789, "000" #tuples, like strings, are immutable sequences. 

In [12]: my_tuple
Out[12]: (123, 456, 789, '000')



\end{lstlisting}
\end{example}

\begin{problem}
\item What is the difference between mutable and immutable objects?
\item If \li{a = ["mushrooms", "rock climbing", 1947, 1954, "yoga"]}
\begin{itemize}
	\item How would you access "yoga"? 
	\item How would you view a copy of the entire list?
	\item How would you clear the entire list? 
	\item How would you find the length? 
	\item How would you assign "mushrooms" (first entry) and "rock climbing" (second entry) 
	of \li{a} to "Peter Pan" and "camelbak"? 
	(This should be done in one line of code and is recognized as a slice assignment)
	\item How would you add "Jonathan, my pet fish" to the end of the list?
\end{itemize}
\item What are the two ways to create sets? Which way must be used to 
create an empty set?
\item What are the two ways to create dictionaries? Which way must be used to
create an empty dictionary?
\item Note that you can create dictionaries using dictionary comprehensions
just like the example above created a list using list comprehension.
Create a dictionary using dictionary comprehension to produce the following output:
\begin{lstlisting}
{2: 4, 4: 16, 6: 36, 8: 64, 10: 100}
\end{lstlisting}
\item How do you delete a key:value pair?
\item How do you access a list of all the keys in your dictionary? And all the values?


\end{problem}


\subsection*{Control Flow Tools}

\begin{example}
Python also supports the usual control flow statements used in other languages
including the \li{while} statement, \li{if} statements, and the definining of functions. 
Control flow tools control the order in which your code is executed. 
\begin{lstlisting}
In [01]: #the Fibonacci series can easily be formulated with a while statement.

In [02]: a, b = 0, 1

In [03]: while b < 10: #while this condition holds, do the following
   ....:     print a
   ....:     a, b = b, a+b #update your variables
   ....:     
0
1
1
2
3
5

In [04]: food = "bagel"

In [05]: if food == "apple":
   ....:     print "72 calories"
   ....: elif food == "banana":
   ....:     print "105 calories"
   ....: elif food == "egg":
   ....:     print "102 calories"
   ....: elif food == "oatmeal":
   ....:     print "147 calories"
   ....: elif food == "pizza":
   ....:     print "298 calories"
   ....: else: 
   ....:     print "calorie count unavailable"
   ....:     
calorie count unavailable

In [06]: ''' note that the else statement is optional, but if used
   ....: needs no condition. '''
   
In [07]: my_list = ["Henry XXI", "beta fish", "asparagus", 4]

In [08]: for i in range(len(my_list)): #the range() function allows iteration over a sequence of numbers. 
   ....:     my_list.append(i)
   ....:     

In [09]: my_list
Out[09]: ['Henry XXI', 'beta fish', 'asparagus', 4, 0, 1, 2, 3]

In [10]: def fibonacci(n): #keyword def introduces a function definition
   ....:     a, b = 0, 1
   ....:     while a < n:
   ....:         print a, b
   ....:         a, b = b, a+b
   ....:         

In [11]: #now the fibonacci function can be called. Try fibonacci(1000)!

In [12]: fibonacci(10) #available to make sure you have the same output.
Out[12]:
0 1
1 1
1 2
2 3
3 5
5 8
8 13

\end{lstlisting}
\end{example}

\begin{problem}
Answer the following questions with the best answer:
\begin{itemize}
\item What is the difference between the print and return statement?
\item What is wrong with the following code?
\begin{lstlisting}
Grocery List = ['pineapple', 'orange juice', "avocados", "pesto sauce"]
for i in range(Grocery List)
if i % 2 = 0
print i, Grocery List(i)
\end{lstlisting}
provided you want the following output:
\begin{lstlisting}
0 pineapple
2 avocados
\end{lstlisting}
\end{itemize}
\end{problem}


\subsection*{Data Structures}
\begin{example}
The list and set data types have many methods. 
\begin{lstlisting}
In [01]: my_list = ["a", "b", "c"]

In [02]: my_list.append("d") #adds "d" to the end of the list

In [03]: my_list.insert(0, 123) #inserts 123 at index 0 in list

In [04]: my_list.remove("c")

In [05]: my_list.sort() #sorts the items of the list in place.

In [06]: my_list.reverse() #reverses the elements of the list in place. 

In [07]: del my_list[3] # deletes the item at index 3.

In [08]: gym_members = set(["Doe, John", "Smith, Jane", "Brown, Bob", "Jones, Sally"])

In [09]: gym_members # note that sets are not ordered by the user
Out[09]: {'Brown, Bob', 'Doe, John', 'Jones, Sally', 'Smith, Jane'}

In [10]: gym_members.add("Lytle, Josh") # sets are very efficient with insertion

In [10]: gym_members.add("Doe, John") # sets do not allow for duplicates

In [11]: gym_members
Out[11]: {'Brown, Bob', 'Doe, John', 'Jones, Sally', 'Lytle, Josh', 'Smith, Jane'}

In [12]: library_members = set(["Lytle, Jane", "Henriksen, Ian", "Smith, Jane", "Grout, Ryan"])

In [13]: library_members.discard("Smith, Jane") # sets are very efficient with removal

In [14]: library_members.add("Lytle, Josh") 

In [15]: library_members
Out[15]: {'Grout, Ryan', 'Henriksen, Ian', 'Lytle, Jane', 'Lytle, Josh'}

In [16]: inter = set.intersection(gym_members, library_members) # set objects also support mathematical operations like union, intersection, difference, and symmetric difference.

In [17]: inter
Out[17]: {'Lytle, Josh'}

In [18]: library_members & gym_members 
Out[18]: {'Lytle, Josh'}


\end{lstlisting}

\end{example}


\begin{problem}
Answer the following questions with the best answer:

\begin{itemize}
\item Implement the following pseudo code: Create an empty list. 
Add 5 integers to your list. 
Cast the integer at index 3 as a float.
Remove the integer at index 2. 
Sort your list backwards. 
\item Implement the following pseudocode: Create two empty sets.
Add 5 integers to the first set and 5 strings to the second set.
Take the union of these sets.

\end{itemize}
\end{problem}

\section*{Specifications}
The following is a suggested format for the \li{solutions.py} file to be submitted.
\begin{lstlisting}
# Problem 1
'''
1. 
2. 
3. 
4. 
5. 
6. 
'''

# Problem 2	
'''
1. 
2.
3. 
4. 
5. 
6. 
7. 
'''

# Problem 3
'''	
1.
2.
'''

# Problem 4
'''
1. 
2. 
'''
\end{lstlisting}	
