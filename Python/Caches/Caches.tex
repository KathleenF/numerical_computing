\lab{Python}{Computer Caches}{Caches}
\objective{Understand the role caches play in computational performance.}

The real computational bottleneck of computers lies in the memory system, not the processor.  In the early days of computing, memory speed and CPU speed were evenly matched.  The CPU relatively quickly access data directly from RAM.  Now, the CPU spends a large amount of time waiting for data to process.  RAM simply isn't fast enough to keep up.


%My processor has
% Level 1 cache size    2 x 32 KB instruction caches
%                       2 x 32 KB data caches
% Level 2 cache size    2 x 256 KB
% Level 3 cache size    4 MB

\begin{problem}
Find the cache line value for your machine.  Plot the results of your timings.
\label{prob:cacheline}
\end{problem}

\begin{problem}
Measure the size of your L1 and L2 caches using the cache line value obtained in
Problem \ref{prob:cacheline}.  Plot the results of your measurements.
\end{problem}

